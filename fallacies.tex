\documentclass[10pt,a4paper,british]{article}
\usepackage[a4paper,width=170mm,top=25mm,bottom=25mm]{geometry}
\usepackage[document]{}
\usepackage[hidelinks,urlcolor=red]{hyperref}
\usepackage{tocloft}
\usepackage{multicol,caption}
\usepackage{lmodern}
\usepackage{textcomp}
\usepackage{url}
\usepackage{xcolor}
\hypersetup{
    colorlinks=true,
    linkcolor=black,
    filecolor=magenta,      
    urlcolor=cyan,
    }
\setlength\parindent{0pt}
\setlength{\parskip}{10pt plus 1pt minus 1pt}
\setcounter{tocdepth}{4}
\setcounter{secnumdepth}{3}
\setlength{\cftsubsecnumwidth}{4em}

\begin{document} 
\thispagestyle{empty} 
\pagenumbering{gobble} 
\begin{center}
\LARGE{\textbf{Master List of Logical Fallacies }}\\ 
	\vspace*{1cm}
	\small{Fallacies are fake or deceptive arguments, ``junk cognition,'' that is,
	arguments that seem irrefutable but prove nothing. Fallacies often seem
	superficially sound and they far too often retain immense persuasive power even
	after being clearly exposed as false. Like epidemics, fallacies sometimes
	``burn through'' entire populations, often with the most tragic results, before
	their power is diminished or lost. Fallacies are not always deliberate, but a
	good scholar’s purpose is always to identify and unmask fallacies in arguments.
	Note that many of these definitions overlap, but the goal here is to identify
	contemporary and classic fallacies as they are used in today's discourse.
	Effort has been made to avoid mere word{-}games (e.g., ``The Fallacist's
	Fallacy,'' or the famous ``Crocodile's Paradox'' of classic times, or the
	so{-}called ``fallacies'' of purely formal and symbolic, business and
	financial, religious or theological logic.  No claim is made to ``academic
	rigor'' in this listing.
	\href{http://utminers.utep.edu/omwilliamson/ENGL1311/fallacies.htm}{Source
	(CC) LICENSE}}
\vspace*{1cm} 
\end{center}

\begin{multicols}{2} 
	\tableofcontents{\normalsize} 
	\newpage
	\pagenumbering{arabic} 
	\setcounter{page}{1} 

	\section{Fallacies}

	\subsection{The A Priori Argument} (also, Rationalization; Dogmatism, Proof
	Texting.) A corrupt argument from logos, starting with a given, pre{-}set
	belief, dogma, doctrine, scripture verse, ``fact'' or conclusion and then
	searching for any reasonable or reasonable{-}sounding argument to
	rationalize, defend or justify it. Certain ideologues and religious
	fundamentalists are proud to use this fallacy as their primary method of
	``reasoning'' and some are even honest enough to say so. E.g., since we
	know there is no such thing as ``evolution,''a prime duty of believers is
	to look for ways to explain away growing evidence, such as is found in DNA,
	that might suggest otherwise. See also the Argument from Ignorance. The
	opposite of this fallacy is the Taboo.

    \subsection{Ableism} (also, The Con Artist's Fallacy; The Dacoit's Fallacy;
    Shearing the Sheeple; Profiteering; ``Vulture Capitalism,'' ``Wealth is
    disease, and I am the cure.'') A corrupt argument from ethos, arguing that
    because someone is intellectually slower, physically or emotionally less
    capable, less ambitious, less aggressive, older or less healthy (or simply
    more trusting or less lucky) than others, s/he ``naturally'' deserves less
    in life and may be freely victimized by those who are luckier, quicker,
    younger, stronger, healthier, greedier, more powerful, less moral or more
    gifted (or who simply have more immediate felt need for money, often
    involving some form of addiction). This fallacy is a ``softer'' argumentum
    ad baculum. When challenged, those who practice this fallacy seem to most
    often shrug their shoulders and mumble ``Life is ruff and you gotta be tuff
    [sic],'' ``You gotta do what you gotta do to get ahead in this world,''
    ``It's no skin off my nose,'' ``That's free enterprise,'' ``That's the way
    life is!'' or similar.

	\subsection{ Actions have Consequences }  The contemporary fallacy of a
	person in power falsely describing an imposed punishment or penalty as a
	``consequence'' of another's negative act. E.g.,`` The consequences of your
	misbehavior could include suspension or expulsion.'' A corrupt argument
	from ethos, arrogating to oneself or to one's rules or laws an ethos of
	cosmic inevitability, i.e., the ethos of God, Fate, Karma, Destiny or
	Reality Itself. Illness or food poisoning are likely ``consequences'' of
	eating spoiled food, while being ``grounded'' is a punishment for, not a
	``consequence,'' of childhood misbehavior. Freezing to death is a natural
	``consequence'' of going out naked in subzero weather but going to prison
	is a punishment for bank robbery, not a natural, inevitable or unavoidable
	``consequence,'' of robbing a bank.  Not to be confused with the Argument
	from Consequences, which is quite different. See also Blaming the Victim.
	An opposite fallacy is that of Moral Licensing.

    \subsection{The Ad Hominem Argument} (also, ``Personal attack,'' ``Poisoning
    the well'') The fallacy of attempting to refute an argument by attacking
    the opposition’s intelligence, morals, education, professional
    qualifications, personal character or reputation, using a corrupted
    negative argument from ethos. E.g., ``That so{-}called judge;'' or ``He's
    so evil that you can't believe anything he says.'' See also ``Guilt by
    Association.'' The opposite of this is the ``Star Power'' fallacy.  Another
    obverse of Ad Hominem is the Token Endorsement Fallacy, where, in the words
    of scholar Lara Bhasin, ``Individual A has been accused of anti{-}Semitism,
    but Individual B is Jewish and says Individual A is not anti{-}Semitic, and
    the implication of course is that we can believe Individual B because,
    being Jewish, he has special knowledge of anti{-} Semitism. Or, a
    presidential candidate is accused of anti{-}Muslim bigotry, but someone
    finds a testimony from a Muslim who voted for said candidate, and this is
    trotted out as evidence against the candidate's bigotry.''  The same
    fallacy would apply to a sports team offensively named after a marginalized
    ethnic group,  but which has obtained the endorsement (freely given or
    paid) of some member, traditional leader or tribal council of that
    marginalized group so that the otherwise{-}offensive team name and logo
    magically become ``okay'' and nonracist.   

	\subsection{The Affective Fallacy} (also The Romantic Fallacy; Emotion over
	Reflection; ``Follow Your Heart'') An extremely common modern fallacy of
	Pathos, that one's emotions, urges or ``feelings'' are innate and in every
	case self{-}validating, autonomous, and above any human intent or act of
	will (one's own or others'), and are thus immune to challenge or criticism.
	(\href{https://www.eurekalert.org/pub_releases/2017-02/nyu-eac021517.php}{In
	fact, researchers now [2017] have robust scientific evidence that emotions
	are actually cognitive and not innate.}) In this fallacy one argues, ``I
	feel it, so it must be true. My feelings are valid, so you have no right to
	criticize what I say or do, or how I say or do it.'' This latter is also a
	fallacy of stasis, confusing a respectful and reasoned response or
	refutation with personal invalidation, disrespect, prejudice, bigotry,
	sexism, homophobia or hostility. A grossly sexist form of the Affective
	Fallacy is the well{-}known crude fallacy that the phallus ``Has no
	conscience'' (also, ``A man's gotta do what a man's gotta do;'' ``Thinking
	with your other head.''), i.e., since (male) sexuality is self{-}validating
	and beyond voluntary control what one does with it cannot be controlled
	either and such actions are not open to criticism, an assertion eagerly
	embraced and extended beyond the male gender in certain reifications of
	``Desire'' in contemporary academic theory. See also, Playing on Emotion.
	Opposite to this fallacy is the Chosen Emotion Fallacy (thanks to scholar
	Marc Lawson for identifying this fallacy), in which one falsely claims
	complete, or at least reliable prior voluntary control over one's own
	autonomic, ``gut level'' affective reactions. Closely related if not
	identical to this last is the ancient fallacy of Angelism, falsely claiming
	that one is capable of ``objective'' reasoning and judgment without
	emotion, claiming for oneself a viewpoint of Olympian  ``disinterested
	objectivity'' or pretending to place oneself far above all personal
	feelings, temptations or bias. See also, Mortification.

	\subsection{ Alphabet Soup } A corrupt modern implicit fallacy from ethos
	in which a person inappropriately overuses acronyms, abbreviations, form
	numbers and arcane insider ``shop talk'' primarily to prove to an audience
	that s/he ``speaks their language'' and is ``one of them'' and to shut out,
	confuse or impress outsiders. E.g., ``It's not uncommon for a K{-}12 with
	ASD to be both GT and LD;'' ``I had a twenty{-}minute DX Q{-}so on 15 with
	a Zed{-}S1 and a couple of LU2's even though the QR{-}Nancy was 10 over
	S9;'' or ``I hope I'll keep on seeing my BAQ on my LES until the day I get
	my DD214.''   See also, Name Calling. This fallacy has recently become
	common in media pharmaceutical advertising in the United States, where
	``Alphabet Soup'' is used to create false identification with and to
	exploit patient groups suffering from specific illnesses or conditions,
	e.g., ``If you have DPC with associated ZL you can keep your B2D under
	control with Luglugmena®. Ask your doctor today about Luglugmena® Helium
	Tetracarbide lozenges to control symptoms of ZL and to keep your B2D under
	that crucial 7.62 threshold. Side effects of  Luglugmena® may include K4
	Syndrome which may lead to lycanthropic bicephaly, BMJ and occasionally,
	death. Do not take Luglugmena® if you are allergic to dogbite or have type
	D Flinder's Garbosis..''.

	\subsection{Alternative Truth} (also, Alt Facts; Counterknowledge;
	Disinformation; Information Pollution) A newly{-}famous contemporary
	fallacy of logos rooted in postmodernism, denying the resilience of facts
	or truth as such. Writer Hannah Arendt, in her
	\href{https://www.amazon.com/Origins-Totalitarianism-Hannah-Arendt/dp/0156701537/}{The
	Origins of Totalitarianism (1951)} warned that ``The ideal subject of
	totalitarian rule is not the convinced Nazi or the dedicated communist, but
	people for whom the distinction between fact and fiction, true and false,
	no longer exists.'' Journalist Leslie Grass (2017) writes in her Blog
	\href{http://www.reachoutrecovery.com/the-recovery-daily/trending-now/presenting-alternate-facts-is-gaslighting}{Reachoutrecovery.com},
	``Is there someone in your life who insists things happened that didn’t
	happen, or has a completely different version of events in which you have
	the facts? It’s a form of mind control and is very common among families
	dealing with substance and behavior problems.'' She suggests that such
	``Alternate Facts'' work to ``put you off balance,'' ``control the story,''
	and ``make you think you're crazy,'' and she notes that ``presenting
	alternate facts is the hallmark of untrustworthy people.''  The Alternative
	Truth fallacy is related to the Big Lie Technique. See also Gaslighting,
	Blind Loyalty, The Big Brain/Little Brain Fallacy, and Two Truths

	\subsection{ The Appeal to Closure } The contemporary fallacy that an
	argument, standpoint, action or conclusion no matter how questionable must
	be accepted as final or else the point will remain unsettled, which is
	unthinkable because those affected will be denied ``closure.'' This fallacy
	falsely reifies a specialized term (closure) from Gestalt Psychology while
	refusing to recognize the undeniable truth that some points will indeed
	remain open and unsettled, perhaps forever. E.g., ``Society would be
	protected, real punishment would be inflicted, crime would be deterred and
	justice served if we sentenced you to life without parole, but we need to
	execute you in order to provide some closure.'' See also, Argument from
	Ignorance, and Argument from Consequences. The opposite of this fallacy is
	the Paralysis of Analysis.

	\subsection{ The Appeal to Heaven } (also, Argumentum ad Coelum, Deus Vult,
	Gott mit Uns, Manifest Destiny, American Exceptionalism, or the Special
	Covenant) An ancient, extremely dangerous fallacy (a deluded argument from
	ethos) that of claiming to know the mind of God (or History, or a higher
	power), who has allegedly ordered or anointed, supports or approves of
	one's own country, standpoint or actions so no further justification is
	required and no serious challenge is possible. (E.g., ``God ordered me to
	kill my children,'' or ``We need to take away your land, since God [or
	Scripture, or Manifest Destiny, or Fate, or Heaven] has given it to us as
	our own.'') A private individual who seriously asserts this fallacy risks
	ending up in a psychiatric ward, but groups or nations who do it are far
	too often taken seriously. Practiced by those who will not or cannot tell
	God's will from their own, this vicious (and blasphemous) fallacy has been
	the cause of endless bloodshed over history. See also, Moral Superiority,
	and Magical Thinking. Also applies to deluded negative Appeals to Heaven,
	e.g., ``You say that famine and ecological collapse due to climate change
	are real dangers during the coming century, but I know God wouldn't ever
	let that happen to us!'' The opposite of the Appeal to Heaven is the Job's
	Comforter fallacy.

	\subsection{The Appeal to Nature} (also, Biologizing; The Green Fallacy)
	The contemporary romantic fallacy of ethos (that of ``Mother Nature'') that
	if something is ``natural'' it has to be good, healthy and beneficial.
	E.g., ``Our premium herb tea is lovingly brewed from the finest
	freshly{-}picked and delicately dried natural T. Radicans leaves. Those who
	dismiss it as mere Poison Ivy'' don't understand that it's 100\% organic,
	with no additives, GMO's or artificial ingredients  It's time to Go Green
	and lay back in Mother's arms.'' One who employs or falls for this fallacy
	forgets the old truism that left to itself, nature is indeed ``red in tooth
	and claw.'' This fallacy also applies to arguments alleging that something
	is ``unnatural,'' or ``against nature'' and thus evil (The Argument from
	Natural Law) e.g. ``Homosexuality should be outlawed because it's against
	nature,'' arrogating to oneself the authority to define what is ``natural''
	and what is unnatural or perverted. E.g., during the American Revolution
	British sources widely condemned rebellion against King George III as
	``unnatural,'' and American revolutionaries as ``perverts,'' because the
	Divine Right of Kings represented Natural Law, and according to 1 Samuel
	1523 in the Bible, rebellion is like unto witchcraft.

	\subsection{The Appeal to Pity } (also, ``Argumentum ad Miserecordiam'') The
	fallacy of urging an audience to “root for the underdog” regardless of the
	issues at hand. A classic example is, “Those poor, cute little squeaky mice
	are being gobbled up by mean, nasty cats ten times their size!” A
	contemporary example might  be America's uncritical popular support for the
	Arab Spring movement of 2010{-}2012 in which The People (``The underdogs'')
	were seen to be heroically overthrowing cruel dictatorships, a movement
	that has resulted in retrospect in chaos, impoverishment, anarchy, mass
	suffering, civil war, the regional collapse of civilization and rise of
	extremism, and the largest refugee crisis since World War II. A corrupt
	argument from pathos. See also, Playing to Emotions. The opposite of the
	Appeal to Pity is the Appeal to Rigor, an argument (often based on machismo
	or on manipulating an audience's fear) based on mercilessness. E.g., ``I'm
	a real man, not like those bleeding hearts, and I'll be tough on [fill in
	the name of the enemy or bogeyman of the hour].''  In academia this latter
	fallacy applies to politically{-}motivated or elitist calls for ``Academic
	Rigor,'' and rage against university developmental / remedial classes, open
	admissions, ``dumbing down'' and ``grade inflation''.

    \subsection{The Appeal to Tradition} (also, Conservative Bias; Back in Those
    Good Times, ``The Good Old Days'') The ancient fallacy that a standpoint,
    situation or action is right, proper and correct simply because it has
    ``always'' been that way, because people have ``always'' thought that way, or
    because it was that way long ago (most often meaning in the audience
    members youth or childhood, not before) and still continues to serve one
    particular group very well. A corrupted argument from ethos (that of past
    generations). E.g., ``In America, women have always been paid less, so
    let's not mess with long{-}standing tradition.''  See also Argument from
    Inertia, and Default Bias. The opposite of this fallacy is The Appeal to
    Novelty (also, ``Pro{-}Innovation bias,'' ``Recency Bias,'' and ``The Bad
    Old Days;'' The Early Adopter's Fallacy), e.g., ``It's NEW, and [therefore
    it must be] improved!'' or ``This is the very latest discovery{-}{-}it has
    to be better''.

	\subsection{Appeasement} (also, ``Assertiveness,'' ``The squeaky wheel gets
	the grease;'' ``I know my rights!'') This fallacy, most often popularly
	connected to the shameful pre{-}World War II appeasement of Hitler, is in
	fact still commonly practiced in public agencies, education and retail
	business today, e.g. ``Customers are always right, even when they're wrong.
	Don't argue with them, just give'em what they want so they'll shut up and
	go away, and not make a stink{-}{-}it's cheaper and easier than a
	lawsuit.''  Widespread unchallenged acceptance of this fallacy encourages
	offensive, uncivil public behavior and sometimes the development of a
	coarse subculture of obnoxious, ``assertive'' manipulators who, like
	``spoiled'' children, leverage their knowledge of how to figuratively (or
	sometimes even literally!) ``make a stink'' into a primary coping skill in
	order to get what they want when they want it. The works of the late
	Community Organizing guru
	\href{https://www.amazon.com/s/ref=dp_byline_sr_book_1?ie=UTF8&text=Saul+Alinsky&search-alias=books&field-author=Saul+Alinsky&sort=relevancerank}{Saul
	Alinsky} suggest practical, nonviolent ways for groups to harness the power
	of this fallacy to promote social change, for good or for evil.. See also
	Bribery.

    \subsection{The Argument from Consequences} (also, Outcome Bias) The major
    fallacy of logos, arguing that something cannot be true because if it were
    the consequences or outcome would be unacceptable. (E.g., ``Global climate
    change cannot be caused by human burning of fossil fuels, because if it
    were, switching to non{-}polluting energy sources would bankrupt American
    industry,'' or ``Doctor, that's wrong! I can't have terminal cancer,
    because if I did that'd mean that I won't live to see my kids get
    married!'') Not to be confused with Actions have Consequences.

	\subsection{The Argument from Ignorance} (also, Argumentum ad Ignorantiam)
	The fallacy that since we don’t know (or can never know, or cannot prove)
	whether a claim is true or false, it must be false, or it must be true.
	E.g., “Scientists are never going to be able to positively prove their
	crazy theory that humans evolved from other creatures, because we weren't
	there to see it! So, that proves the Genesis six{-}day creation account is
	literally true as written!” This fallacy includes Attacking the Evidence
	(also, ``Whataboutism''; The Missing Link fallacy), e.g. ``Some or all of
	your key evidence is missing, incomplete, or even faked!  What about that?
	That proves you're wrong and I'm right!'' This fallacy usually includes
	fallacious “Either{-}Or Reasoning” as well E.g., “The vet can't find any
	reasonable explanation for why my dog died. See! See! That proves that you
	poisoned him! There’s no other logical explanation!” A corrupted argument
	from logos, and a fallacy commonly found in American political, judicial
	and forensic reasoning. The recently famous ``Flying Spaghetti Monster''
	meme is a contemporary refutation of this fallacy{-}{-}simply because we
	cannot conclusively disprove the existence of such an absurd entity does
	not argue for its existence. See also A Priori Argument, Appeal to Closure,
	The Simpleton's Fallacy, and Argumentum ex Silentio.

    \subsection{The Argument from Incredulity} The popular fallacy of doubting or
    rejecting a novel claim or argument out of hand simply because it appears
    superficially ``incredible,'' ``insane'' or ``crazy,'' or because it goes
    against one's own personal beliefs, prior experience or ideology.  This
    cynical fallacy falsely elevates the saying popularized by Carl Sagan, that
    ``Extraordinary claims require extraordinary proof,'' to an absolute law of
    logic. See also Hoyle's Fallacy. The common, popular{-}level form of this
    fallacy is dismissing surprising, extraordinary or unfamiliar arguments and
    evidence with a wave of the hand, a shake of the head, and a mutter of
    ``that's crazy''!

    \subsection{The Argument from Inertia} (also “Stay the Course”) The fallacy
    that it is necessary to continue on a mistaken course of action regardless
    of pain and sacrifice involved  and even after discovering it is mistaken,
    because changing course would mean admitting that one's decision (or one's
    leader, or one's country, or one's faith) was wrong, and all one's effort,
    expense, sacrifice and even bloodshed was for nothing, and that's
    unthinkable. A variety of the Argument from Consequences, E for Effort, or
    the Appeal to Tradition. See also ``Throwing Good Money After Bad''.

	\subsection{The Argument from Motives} (also Questioning Motives) The
	fallacy of declaring a standpoint or argument invalid solely because of the
	evil, corrupt or questionable motives of the one making the claim. E.g.,
	``Bin Laden wanted us to withdraw from Afghanistan, so we have to keep up
	the fight!'' Even evil people with the most corrupt motives sometimes say
	the truth (and even good people with the highest and purest motives are
	often wrong or mistaken). A variety of the Ad Hominem argument. The
	opposite side of this fallacy is falsely justifying or excusing evil or
	vicious actions because of the perpetrator's aparent purity of motives or
	lack of malice.  (E.g., ``Sure, she may have beaten her children bloody now
	and again but she was a highly educated, ambitious professional woman at
	the end of her rope, deprived of adult conversation and stuck between four
	walls for years on end with a bunch of screaming, fighting brats, doing the
	best she could with what little she had. How can you stand there and accuse
	her of child abuse?'') See also Moral Licensing.

	\subsection{Argumentum ad Baculum} (``Argument from the Club.'' Also,
	``Argumentum ad Baculam,'' ``Argument from Strength,'' ``Muscular
	Leadership,'' ``Non{-}negotiable Demands,'' ``Hard Power,'' Bullying, The
	Power{-}Play, Fascism, Resolution by Force of Arms, Shock and Awe.) The
	fallacy of ``persuasion'' or ``proving one is right'' by force, violence,
	brutality, terrorism, superior strength, raw military might, or threats of
	violence. E.g., ``Gimmee your wallet or I'll knock your head off!'' or ``We
	have the perfect right to take your land, since we have the big guns and
	you don't.'' Also applies to indirect forms of threat. E.g., ``Give up your
	foolish pride, kneel down and accept our religion today if you don't want
	to burn in hell forever and ever!'' A mainly discursive Argumentum ad
	Baculum is that of forcibly silencing opponents, ruling them ``out of
	order,'' blocking, censoring or jamming their message, or simply speaking
	over them or/speaking more loudly than they do, this last a tactic
	particularly attributed to men in mixed{-}gender discussions.

	\subsection{Argumentum ad Mysteriam} (``Argument from Mystery;'' also
	Mystagogy.) A darkened chamber, incense, chanting or drumming, bowing and
	kneeling, special robes or headgear, holy rituals and massed voices
	reciting sacred mysteries in an unknown tongue  have a quasi{-}hypnotic
	effect and can often persuade more strongly than any logical argument.  The
	Puritan Reformation was in large part a rejection of this fallacy. When
	used knowingly and deliberately this fallacy is particularly vicious and
	accounts for some of the fearsome persuasive power of cults.  An example of
	an Argumentum ad Mysteriam is the ``Long Ago and Far Away'' fallacy, the
	fact that facts, evidence, practices or arguments from ancient times,
	distant lands and/or ``exotic'' cultures  seem to acquire a special
	gravitas or ethos simply because of their antiquity, language or origin,
	e.g., publicly chanting Holy Scriptures in their original (most often
	incomprehensible) ancient languages, preferring the Greek, Latin, Assyrian
	or Old Slavonic Christian Liturgies over their vernacular versions, or
	using classic or newly invented Greek and Latin names for fallacies in
	order to support their validity. See also, Esoteric Knowledge. An obverse
	of the Argumentum ad Mysteriam is the Standard Version Fallacy.

	\subsection{Argumentum ex Silentio} (Argument from Silence) The fallacy
	that if available sources remain silent or current knowledge and evidence
	can prove nothing about a given subject or question this fact in itself
	proves the truth of one's claim. E.g., ``Science can tell us nothing about
	God.  That proves God doesn't exist.'' Or ``Science admits it can tell us
	nothing about God, so you can't deny that God exists!'' Often misused in
	the American justice system, where, contrary to the 5th Amendment and the
	legal presumption of innocence until proven guilty,  remaining silent or
	``taking the Fifth'' is often falsely portrayed as proof of guilt. E.g.,
	``Mr. Hixon can offer no alibi for his whereabouts the evening of January
	15th. This proves that he was in fact in room 331 at the Smuggler's Inn,
	murdering his wife with a hatchet!'' In today's America, choosing to remain
	silent in the face of a police officer's questions can make one guilty
	enough to be arrested or even shot. See also, Argument from Ignorance.

	\subsection{Availability Bias} (also, Attention Bias, Anchoring Bias) A
	fallacy of logos stemming from the natural tendency to give undue attention
	and importance to information that is immediately available at hand,
	particularly the first or last information received, and to minimize or
	ignore broader data or wider evidence that clearly exists but is not as
	easily remembered or accessed. E.g., ``We know from experience that this
	doesn't work,'' when ``experience'' means the most recent local attempt,
	ignoring overwhelming experience from other places and times where it has
	worked and does work. Also related is the fallacy of Hyperbole [also,
	Magnification, or sometimes Catastrophizing] where an immediate instance is
	immediately proclaimed ``the most significant in all of human history,'' or
	the ``worst in the whole world!'' This latter fallacy works extremely well
	with less{-}educated audiences and those whose ``whole world'' is very
	small indeed, audiences who ``hate history'' and whose historical memory
	spans several weeks at best.

	\subsection{The Bandwagon Fallacy} (also, Argument from Common Sense,
	Argumentum ad Populum) The fallacy of arguing that because ``everyone,''
	``the people,'' or ``the majority'' (or someone in power who has widespread
	backing) supposedly thinks or does something, it must therefore be true and
	right. E.g., ``Whether there actually is large scale voter fraud in America
	or not, many people now think there is and that makes it so.'' Sometimes
	also includes Lying with Statistics, e.g. “Over 75\% of Americans believe
	that crooked Bob Hodiak is a thief, a liar and a pervert. There may not be
	any evidence, but for anyone with half a brain that conclusively proves
	that Crooked Bob should go to jail! Lock him up! Lock him up!” This is
	sometimes combined with the ``Argumentum ad Baculum,'' e.g., ``Like it or
	not, it's time to choose sides Are you going to get on board  the bandwagon
	with everyone else, or get crushed under the wheels as it goes by?'' Or in
	the 2017 words of former White House spokesperson Sean Spicer, ``They
	should either get with the program or they can go,'' A contemporary digital
	form of the Bandwagon Fallacy is the Information Cascade, ``in which people
	echo the opinions of others, usually online, even when their own opinions
	or exposure to information contradicts that opinion. When information
	cascades form a pattern, this pattern can begin to overpower later opinions
	by making it seem as if a consensus already exists.'' (Thanks to
	\href{https://www.tolerance.org/magazine/fall-2017/speaking-of-digital-literacy}{Teaching
	Tolerance for this definition!}) See also Wisdom of the Crowd, and The Big
	Lie Technique. For the opposite of this fallacy see the Romantic Rebel
	fallacy. 

	\subsection{The Big Brain/Little Brain Fallacy} (also, the Führerprinzip;
	Mad Leader Disease) A not{-}uncommon but extreme example of the Blind
	Loyalty Fallacy below, in which a tyrannical boss, military commander, or
	religious or cult{-}leader tells followers ``Don't think with your little
	brains (the brain in your head), but with your BIG brain (mine).'' This
	last is sometimes expressed in positive terms, i.e., ``You don't have to
	worry and stress out about the rightness or wrongness of what you are doing
	since I, the Leader. am assuming all moral and legal responsibility for all
	your actions. So long as you are faithfully following orders without
	question I will defend you and gladly accept all the consequences up to and
	including eternal damnation if I'm wrong.'' The opposite of this is the
	fallacy of ``Plausible Deniability.'' See also, ``Just Do It!'', and
	``Gaslighting''.

	\subsection{The Big ``But'' Fallacy} (also, Special Pleading)  The fallacy
	of enunciating a generally{-}accepted principle and then directly negating
	it with a ``but.'' Often this takes the form of the ``Special Case,'' which
	is supposedly exempt from the usual rules of law, logic, morality, ethics
	or even credibility  E.g., ``As Americans we have always believed on
	principle that every human being has God{-}given, inalienable rights to
	life, liberty and the pursuit of happiness, including in the case of
	criminal accusations a fair and speedy trial before a jury of one's peers.
	BUT, your crime was so unspeakable and a trial would be so problematic for
	national security that it justifies locking you up for life in Guantanamo
	without trial, conviction or possibility of appeal.''  Or, ``Yes, Honey, I
	still love you more than life itself, and I know that in my wedding vows I
	promised before God that I'd forsake all others and be faithful to you
	'until death do us part,' but you have to understand, this was a special
	case...''  See also, ``Shopping Hungry,'' and ``We Have to do Something''!

	\subsection{The Big Lie Technique} (also the Bold Faced Lie; ``Staying on
	Message.'') The contemporary fallacy of repeating a lie, fallacy, slogan,
	talking{-}point, nonsense{-}statement or deceptive half{-}truth over and
	over in different forms (particularly in the media) until it becomes part
	of daily discourse and people accept it without further proof or evidence.
	Sometimes the bolder and more outlandish the Big Lie becomes the more
	credible it seems to a willing, most often angry audience. E.g., ``What
	about the Jewish Problem?'' Note that when this particular phony debate was
	going on there was no ``Jewish Problem,'' only a Nazi Problem, but hardly
	anybody in power recognized or wanted to talk about that, while far too
	many ordinary Germans were only too ready to find a convenient scapegoat to
	blame for their suffering during the Great Depression. Writer Miles J.
	Brewer expertly demolishes The Big Lie Technique in his classic (1930)
	short story,
	\href{https://en.wikisource.org/wiki/Avon_Fantasy_Reader/Issue_10/The_Gostak_and_the_Doshes}{``The
	Gostak and the Doshes.''} However, more contemporary examples of the Big
	Lie fallacy might be the completely fictitious August 4, 1964 ``Tonkin Gulf
	Incident'' concocted under Lyndon Johnson as a false justification for
	escalating the Vietnam War, or the non{-}existent ``Weapons of Mass
	Destruction'' in Iraq (conveniently abbreviated ``WMD's'' in order to lend
	this Big Lie a legitimizing, military{-}sounding ``Alphabet Soup'' ethos),
	used in 2003 as a false justification for the Second Gulf War. The
	November, 2016 U.S. President{-}elect's statement that ``millions'' of
	ineligible votes were cast in that year's American. presidential election
	appears to be a classic Big Lie. See also, Alternative Truth; The Bandwagon
	Fallacy, the Straw Man, Alphabet Soup, and Propaganda.   

	\subsection{Blind Loyalty} (also Blind Obedience, Unthinking Obedience, the
	``Team Player'' appeal, the Nuremberg Defense) The dangerous fallacy that
	an argument or action is right simply and solely because a respected leader
	or source (a President, expert, one’s parents, one's own ``side,'' team or
	country, one’s boss or commanding officers) says it is right. This is
	over{-}reliance on authority, a gravely corrupted argument from ethos that
	puts loyalty above truth,  above one's own reason and above conscience. In
	this case a person attempts to justify incorrect, stupid or criminal
	behavior by whining ``That's what I was told to do,'' or ``I was just
	following orders.''  See also, The Big Brain/Little Brain Fallacy, and The
	``Soldier's Honor'' Fallacy.

	\subsection{Blood is Thicker than Water} (also Favoritism; Compadrismo;
	``For my friends, anything.'') The reverse of the ``Ad Hominem'' fallacy, a
	corrupt argument from ethos where a statement, argument or action is
	automatically regarded as true, correct and above challenge because one is
	related to, knows and likes, or is on the same team or side, or belongs to
	the same religion, party, club or fraternity as the individual involved.
	(E.g., ``My brother{-}in{-}law says he saw you goofing off on the job.
	You're a hard worker but who am I going to believe, you or him? You're
	fired!'')  See also the Identity Fallacy.

	\subsection{Brainwashing} (also, Propaganda, ``Radicalization.'') The Cold
	War{-}era fantasy that an enemy can instantly win over or ``radicalize'' an
	unsuspecting audience with their vile but somehow unspeakably persuasive
	``propaganda,''  e.g., ``Don't look at that website! They're trying to
	brainwash you with their propaganda!'' Historically, ``brainwashing''
	refers more properly to the inhuman Argumentum ad Baculum of  ``beating an
	argument into'' a prisoner via a combination of pain, fear, sensory or
	sleep deprivation, prolonged abuse and sophisticated psychological
	manipulation (also, the ``Stockholm Syndrome.''). Such ``brainwashing'' can
	also be accomplished by pleasure (``Love Bombing,''); e.g., ``Did you like
	that? I know you did. Well, there's lots more where that came from when you
	sign on with us!'' (See also, ``Bribery.'') An unspeakably sinister form of
	persuasion by brainwashing involves deliberately addicting a person to
	drugs and then providing or withholding the substance depending on the
	addict's compliance. Note Only the other side brainwashes. ``We'' never
	brainwash.

	\subsection{Bribery} (also, Material Persuasion, Material Incentive,
	Financial Incentive). The fallacy of ``persuasion'' by bribery, gifts or
	favors is the reverse of the Argumentum ad Baculum. As is well known,
	someone who is persuaded by bribery rarely ``stays persuaded'' in the long
	term unless the bribes keep on coming in and increasing with time. See also
	Appeasement.

	\subsection{Calling ``Cards''} A contemporary fallacy of logos, arbitrarily
	and falsely dismissing familiar or easily{-}anticipated but valid, reasoned
	objections to one's standpoint with a wave of the hand, as mere ``cards''
	in some sort of ``game'' of rhetoric, e.g. ``Don't try to play the `Race
	Card' against me,'' or ``She's playing the `Woman Card' again,'' or ``That
	`Hitler Card' won't score with me in this argument.'' See also, The Taboo,
	and Political Correctness.

    \subsection{Circular Reasoning} (also, The Vicious Circle; Catch 22, Begging
    the Question, Circulus in Probando) A fallacy of logos where A is because
    of B, and B is because of A, e.g., ``You can't get a job without
    experience, and you can't get experience without a job.'' Also refers to
    falsely arguing that something is true by repeating the same statement in
    different words. E.g., “The witchcraft problem is the most urgent spiritual
    crisis in the world today. Why? Because witches threaten our very eternal
    salvation.” A corrupt argument from logos. See also the ``Big Lie
    technique.''

	\subsection{The Complex Question} The contemporary fallacy of demanding a
	direct answer to a question that cannot be answered without first analyzing
	or challenging the basis of the question itself. E.g., ``Just answer me
	`yes' or `no'  Did you think you could get away with plagiarism and not
	suffer the consequences?`` Or, ``Why did you rob that bank?'' Also applies
	to situations where one is forced to either accept or reject complex
	standpoints or propositions containing both acceptable and unacceptable
	parts. A corruption of the argument from logos. A counterpart of Either/Or
	Reasoning.

	\subsection{Confirmation Bias} A fallacy of logos, the common tendency to
	notice, search out, select and share evidence that confirms one's own
	standpoint and beliefs, as opposed to contrary evidence. This fallacy is
	how ``fortune tellers'' work{-}{-}If I am told I will meet a ``tall, dark
	stranger'' I will be on the lookout for a tall, dark stranger, and when I
	meet someone even marginally meeting that description I will marvel at the
	correctness of the ``psychic's'' prediction. In contemporary times
	Confirmation Bias is most often seen in the tendency of various audiences
	to ``curate their political environments, subsisting on one{-}sided
	information diets and [even] selecting into politically homogeneous
	neighborhoods''
	\href{http://science.sciencemag.org/content/355/6328/914.full}{(Michael A.
	Neblo et al., 2017, Science magazine)}.  Confirmation Bias (also,
	Homophily) means that people tend to seek out and follow solely those media
	outlets that confirm their common ideological and cultural biases,
	sometimes to an degree that leads a the false (implicit or even explicit)
	conclusion that ``everyone'' agrees with that bias and that anyone who
	doesn't is ``crazy'', ``looney,'' evil or even ``radicalized.'' See also,
	``Half Truth,'' and ``Defensiveness.'' 

	\subsection{Cost Bias} A fallacy of ethos (that of a product), the fact
	that something expensive (either in terms of money, or something that is
	``hard fought'' or ``hard won'' or for which one ``paid dearly'') is
	generally valued more highly than something obtained free or cheaply,
	regardless of the item's real quality, utility or true value to the
	purchaser. E. g., ``Hey, I worked hard to get this car!  It may be nothing
	but a clunker that can't make it up a steep hill, but it's mine, and to me
	it's better than some millionaire's limo.''  Also applies to judging the
	quality of a consumer item
	\href{https://www.eurekalert.org/pub_releases/2017-09/sfcp-wjy092917.php}{(or
	even of its owner!)} primarily by the item's brand, price, label or source,
	e.g., ``Hey, you there in the Jay{-}Mart suit! Har{-}har!'' or, ``Ooh,
	she's driving a Mercedes''!

	\subsection{Default Bias} (also, Normalization of Evil, ``Deal with it;''
	``If it ain't broke, don't fix it;'' Acquiescence; ``Making one's peace
	with the situation;'' ``Get used to it;'' ``Whatever is, is right;''  ``It
	is what it is;'' ``Let it be, let it be;'' ``This is the best of all
	possible worlds [or, the only possible world];'' ``Better the devil you
	know than the devil you don't.'') The logical fallacy of automatically
	favoring or accepting a situation simply because it exists right now, and
	arguing that any other alternative is mad, unthinkable, impossible, or at
	least would take too much effort, expense, stress or risk to change. The
	opposite of this fallacy is that of Nihilism (``Tear it all down!''),
	blindly rejecting what exists in favor of what could be, the adolescent
	fantasy of romanticizing anarchy, chaos (an ideology sometimes called
	political ``Chaos Theory''), disorder, ``permanent revolution,'' or change
	for change's sake.

    \subsection{Defensiveness} (also, Choice{-}support Bias Myside Bias) A
    fallacy of ethos (one's own), in which after one has taken a given
    decision, commitment or course of action, one automatically tends to defend
    that decision and to irrationally dismiss opposing options even when one's
    decision later on proves to be shaky or wrong. E.g., ``Yeah, I voted for
    Snith. Sure, he turned out to be a crook and a liar and he got us into war,
    but I still say that at that time he was better than the available
    alternatives!''  See also ``Argument from Inertia'' and ``Confirmation
    Bias''.

	\subsection{Deliberate Ignorance} (also, Closed{-}mindedness; ``I don't
	want to hear it!''; Motivated Ignorance; Tuning Out; Hear No Evil, See No
	Evil, Speak No Evil [The Three Monkeys' Fallacy]) As described by author
	and commentator
	\href{https://www.vox.com/policy-and-politics/2017/5/18/15659394/trump-supporters-motivated-ignorance}{Brian
	Resnik on Vox.com} (2017), this is the fallacy of simply choosing not to
	listen, ``tuning out'' or turning off any information, evidence or
	arguments that challenge one's beliefs, ideology, standpoint, or peace of
	mind, following the popular humorous dictum ``Don't try to confuse me with
	the facts; my mind is made up!'' This seemingly innocuous fallacy has
	enabled the most vicious tyrannies and abuses over history, and continues
	to do so today. See also Trust your Gut, Confirmation Bias, The Third
	Person Effect, ``They're All Crooks,'' the Simpleton's Fallacy, and The
	Positive Thinking Fallacy.

    \subsection{Diminished Responsibility} The common contemporary fallacy of
    applying a specialized judicial concept (that criminal punishment should be
    less if one's judgment was impaired) to reality in general. E.g., ``You
    can't count me absent on Monday{-}{-}I was hung over and couldn't come to
    class so it's not my fault.''  Or, ``Yeah, I was speeding on the freeway
    and killed a guy, but I was buzzed out of my mind and didn't know what I
    was doing so it didn't matter that much.'' In reality the death does matter
    very much to the victim, to his family and friends and to society in
    general. Whether the perpetrator was high or not does not matter at all
    since the material results are the same. This also includes the fallacy of
    Panic, a very common contemporary fallacy that one's words or actions, no
    matter how damaging or evil, somehow don't ``count'' because ``I panicked!''
    This fallacy is rooted in the confusion of ``consequences'' with
    ``punishment.''  See also ``Venting''.

	\subsection{Disciplinary Blinders} A very common contemporary scholarly or
	professional fallacy of ethos (that of one's discipline, profession or
	academic field),  automatically disregarding, discounting or ignoring a
	priori otherwise{-}relevant research, arguments and evidence that come from
	outside one's own professional discipline, discourse community or academic
	area of study. E.g., ``That might be relevant or not, but it's so not what
	we're doing in our field right now.''  See also, ``Star Power'' and ``Two
	Truths.'' An analogous fallacy is that of Denominational Blinders,
	arbitrarily ignoring or waving aside without serious consideration any
	arguments or discussion about faith, morality, ethics, spirituality, the
	Divine or the afterlife that come from outside one's own specific religious
	denomination or faith tradition.

    \subsection{Dog{-}Whistle Politics} An extreme version of reductionism and
    sloganeering in the public sphere, a contemporary fallacy of logos and
    pathos in which a brief phrase or slogan of the hour, e.g., ``Abortion,''
    ``The 1\%,'' ``9/11,'' ``Zionism,''``Chain Migration,'' ``Islamic
    Terrorism,'' ``Fascism,'' ``Communism,'' ``Big government,'' ``Taco
    trucks!'', ``Tax and tax and spend and spend,'' ``Gun violence,'' ``Gun
    control,'' ``Freedom of choice,'' ``Lock 'em up,'', ``Amnesty,'' etc. is
    flung out as ``red meat'' or ``chum in the water'' that reflexively sends
    one's audience into a snapping, foaming{-}at{-}the{-}mouth
    feeding{-}frenzy. Any reasoned attempt to more clearly identify,
    deconstruct or challenge an opponent's ``dog whistle'' appeal results in
    puzzled confusion at best and wild, irrational fury at worst. ``Dog
    whistles'' differ widely in different places, moments and cultural milieux,
    and they change and lose or gain power so quickly that even recent historic
    texts sometimes become extraordinarily difficult to interpret. A common but
    sad instance of the fallacy of Dog Whistle Politics is that of  candidate
    ``debaters'' of differing political shades simply blowing a succession of
    discursive ``dog whistles'' at their audience instead of addressing,
    refuting or even bothering to listen to each other's arguments, a situation
    resulting in contemporary (2017) allegations that the political Right and
    Left in America are speaking ``different languages'' when they are simply
    blowing different ``dog whistles.'' See also, Reductionism..

	\subsection{The ``Draw Your Own Conclusion'' Fallacy} (also the
	Non{-}argument Argument; Let the Facts Speak for Themselves). In this
	fallacy of logos an otherwise uninformed audience is presented with
	carefully selected and groomed, ``shocking facts'' and then prompted to
	immediately ``draw their own conclusions.'' E.g., ``Crime rates are more
	than twice as high among middle{-}class Patzinaks than among any other
	similar population group{-}{-}draw your own conclusions.'' It is well known
	that those who are allowed to ``come to their own conclusions'' are
	generally much more strongly convinced than those who are given both
	evidence and conclusion up front. However, Dr. William Lorimer points out
	that ``The only rational response to the non{-}argument is 'So what?' i.e.
	'What do you think you've proved, and why/how do you think you've proved
	it?''' Closely related (if not identical) to this is the well{-}known
	``Leading the Witness'' Fallacy, where a sham, sarcastic or biased question
	is asked solely in order to evoke a desired answer.

	\subsection{The Dunning{-}Kruger Effect} A cognitive bias that leads people
	of limited skills or knowledge to mistakenly believe their abilities are
	greater than they actually are. (Thanks to
	\href{https://www.tolerance.org/magazine/fall-2017/speaking-of-digital-literacy}{Teaching
	Tolerance} for this definition!)  E.g., ``I know Washington was the Father
	of His Country and never told a lie, Pocahontas was the first Native
	American, Lincoln freed the slaves, Hitler murdered six million Jews, Susan
	B. Anthony won equal rights for women, and Martin Luther King said `I have
	a dream!'  Moses parted the Red Sea, Caesar said `Et tu, Brute?' and the
	only reason America didn't win the Vietnam War hands{-}down like we always
	do was because they tied our general's hands and the politicians cut and
	run. See?  Why do I need to take a history course? I know everything about
	history''!

	\subsection{E for Effort}. (also Noble Effort; I'm Trying My Best; The Lost
	Cause) The common contemporary fallacy of ethos that something must be
	right, true, valuable, or worthy of respect and honor solely because one
	(or someone else) has put so much sincere good{-}faith effort or even
	sacrifice and bloodshed into it. (See also Appeal to Pity; Argument from
	Inertia; Heroes All; or Sob Story).  An extreme example of this fallacy is
	Waving the Bloody Shirt (also, the ``Blood of the Martyrs'''Fallacy), the
	fallacy that a cause or argument, no matter how questionable or
	reprehensible, cannot be questioned without dishonoring the blood and
	sacrifice of those who died so nobly for that cause. E.g.,
	\href{https://en.wikipedia.org/wiki/Maryland,_My_Maryland#Lyrics}{``Defend
	the patriotic gore / That flecked the streets of Baltimore...''} (from the
	official Maryland State Song). See also Cost Bias, The Soldier's Honor
	Fallacy, and the Argument from Inertia.

	\subsection{Either/Or Reasoning} (also False Dilemma, All or Nothing
	Thinking; False Dichotomy, Black/White Fallacy, False Binary) A fallacy of
	logos that falsely offers only two possible options even though a broad
	range of possible alternatives, variations and combinations are always
	readily available. E.g., ``Either you are 100\% Simon Straightarrow or you
	are as queer as a three dollar bill{-}{-}it's as simple as that and there's
	no middle ground!'' Or, ``Either you’re in with us all the way or you’re a
	hostile and must be destroyed!  What's it gonna be?''  Or, if your
	performance is anything short of perfect, you consider yourself an abject
	failure. Also applies to falsely contrasting one option or case to another
	that is not really opposed, e.g., falsely opposing ``Black Lives Matter''
	to ``Blue Lives Matter'' when in fact not a few police officers are
	themselves African American, and African Americans and police are not (or
	ought not to be!) natural enemies. Or, falsely posing a choice of either
	helping needy American veterans or helping needy foreign refugees, when in
	fact in today's United States there are ample resources available to easily
	do both should we care to do so.  See also, Overgeneralization.

	\subsection{Equivocation} The fallacy of deliberately failing to define
	one's terms, or knowingly and deliberately using words in a different sense
	than the one the audience will understand. (E.g., President Bill Clinton
	stating that he did not have sexual relations with ``that woman,'' meaning
	no sexual penetration, knowing full well that the audience will understand
	his statement as ``I had no sexual contact of any kind with that woman.'')
	This is a corruption of the argument from logos, and a tactic often used in
	American jurisprudence.  Historically, this referred to a tactic used
	during the Reformation{-}era religious wars in Europe, when people were
	forced to swear loyalty to one or another side and did as demanded via
	``equivocation,''  i.e., ``When I solemnly swore true faith and allegiance
	to the King I really meant to King Jesus, King of Kings, and not to the
	evil usurper squatting on the throne today.'' This latter form of fallacy
	is excessively rare today when the swearing of oaths has become effectively
	meaningless except as obscenity or as speech formally subject to perjury
	penalties in legal or judicial settings.

    \subsection{The Eschatological Fallacy} The ancient fallacy of arguing,
    ``This world is coming to an end, so...''  Popularly refuted by the
    observation that ``Since the world is coming to an end you won't need your
    life savings anyhow, so why not give it all to me''?

	\subsection{Esoteric Knowledge} (also Esoteric Wisdom; Gnosticism; Inner
	Truth; the Inner Sanctum; Need to Know) A fallacy from logos and ethos,
	that there is some knowledge reserved only for the Wise, the Holy or the
	Enlightened, (or those with proper Security Clearance), things that the
	masses cannot understand and do not deserve to know, at least not until
	they become wiser, more trusted or more ``spiritually advanced.''  The
	counterpart of this fallacy is that of Obscurantism (also Obscurationism,
	or Willful Ignorance), that (almost always said in a basso profundo voice)
	``There are some things that we mere mortals must never seek to know!''
	E.g., ``Scientific experiments that violate the privacy of the marital bed
	and expose  the deep and private mysteries of human sexual behavior to the
	harsh light of science are obscene, sinful and morally evil. There are some
	things that we as humans are simply not meant to know!'' For the opposite
	of this latter, see the ``Plain Truth Fallacy.'' See also, Argumentum ad
	Mysteriam.

    \subsection{Essentializing} A fallacy of logos that proposes a person or
    thing “is what it is and that’s all that it is,” and at its core will
    always be the way it is right now (E.g., ``All terrorists are monsters, and
    will still be terrorist monsters even if they live to be 100,'' or ``'The
    poor you will always have with you, so any effort to eliminate poverty is
    pointless.''). Also refers to the fallacy of arguing that something is a
    certain way ``by nature,'' an empty claim that no amount of proof can
    refute. (E.g., ``Americans are cold and greedy by nature,'' or ``Women are
    naturally better cooks than men.'') See also ``Default Bias.''  The
    opposite of this is Relativizing, the typically postmodern fallacy of
    blithely dismissing any and all arguments against one's standpoint by
    shrugging one's shoulders and responding `` Whatever..., I don't feel like
    arguing about it;'' ``It all depends...;'' ``That's your opinion;
    everything's relative;'' or falsely invoking Einstein's Theory of
    Relativity, Heisenberg's Uncertainty Principle, Quantum Weirdness or the
    Theory of Multiple Universes in order to confuse, mystify or ``refute'' an
    opponent. See also, ``Red Herring'' and  ``Appeal to Nature''.

	\subsection{The Etymological Fallacy} (also, ``The Underlying Meaning'') A
	fallacy of logos, drawing false conclusions from the (most often
	long{-}forgotten) linguistic origins of a current word, or the alleged
	meanings or associations of that word in another language. E.g., ``As used
	in physics, electronics and electrical engineering the term ``hysteresis''
	is grossly sexist since it originally came from the Greek word for
	``uterus'' or ``womb.'''  Or, ``I refuse to eat fish! Don't you know that
	the French word for `fish' is `poisson,' which looks just like the English
	word `poison'?  And doesn't that suggest something to you?'' Famously,
	postmodern philosopher Jacques Derrida played on this fallacy at great
	length in his (1968)
	\href{https://www.amazon.com/Derrida-Platos-Pharmacy-ICG-Academic-ebook/dp/B00WQ5P9SU/}{``Plato's
	Pharmacy''}.

	\subsection{The Excluded Middle} A corrupted argument from logos that
	proposes that since a little of something is good, more must be better (or
	that if less of something is good, none at all is even better). E.g., ``If
	eating an apple a day is good for you, eating an all{-}apple diet is even
	better!'' or ``If a low fat diet prolongs your life, a no{-}fat diet should
	make you live forever!''  An opposite of this fallacy is that of Excluded
	Outliers, where one arbitrarily discards evidence, examples or results that
	disprove one's standpoint by simply describing them as ``Weird,''
	``Outliers,'' or ``Atypical.'' See also, ``The Big `But' Fallacy.'' Also
	opposite is the Middle of the Road Fallacy (also, Falacia ad Temperantiam;
	``The Politics of the Center;'' Marginalization of the Adversary), where
	one demonstrates the ``reasonableness'' of one's own standpoint (no matter
	how extreme) not on its own merits, but solely or mainly by presenting it
	as the only ``moderate'' path among two or more obviously unacceptable
	extreme alternatives.  E.g., anti{-}Communist scholar
	\href{https://www.amazon.fr/grammaire-politique-lenine-CHARLES-ROIG/dp/B005WQHW3K/}{Charles
	Roig (1979)} notes that Vladimir Lenin successfully argued for Bolshevism
	in Russia as the only available ``moderate'' middle path between
	bomb{-}throwing Nihilist terrorists on the ultra{-}left and a corrupt and
	hated Czarist autocracy on the right. As Texas politician and humorist Jim
	Hightower famously
	\href{https://www.brainyquote.com/quotes/quotes/j/jimhightow135322.html}{declares}
	in an undated quote, ``The middle of the road is for yellow lines and dead
	armadillos''.

	\subsection{The ``F{-}Bomb'' } (also Cursing; Obscenity; Profanity). An
	adolescent fallacy of pathos, attempting to defend or strengthen one's
	argument with gratuitous, unrelated sexual, obscene, vulgar, crude or
	profane language when such language does nothing to make an argument
	stronger, other than perhaps to create a sense of identity with certain
	young male ``urban'' audiences. This fallacy also includes adding
	gratuitous sex scenes or ``adult'' language to an otherwise unrelated novel
	or movie, sometimes simply to avoid the dreaded ``G'' rating. Related to
	this fallacy is the Salacious Fallacy, falsely attracting attention to and
	thus potential agreement with one's argument by inappropriately sexualizing
	it, particularly connecting it to some form of sex that is perceived as
	deviant, perverted or prohibited (E.g., Arguing against Bill Clinton's
	presidential legacy by continuing to wave Monica's Blue Dress, or against
	Donald Trump's presidency by obsessively highlighting his past boasting
	about genital groping). Historically, this dangerous fallacy was deeply
	implicated with the crime of lynching, in which false, racist accusations
	against a Black or minority victim were almost always salacious in nature
	and the sensation involved was successfully used to whip up public emotion
	to a murderous pitch. See also, Red Herring.

	\subsection{The False Analogy} The fallacy of incorrectly comparing one
	thing to another in order to draw a false conclusion. E.g., ``Just like an
	alley cat needs to prowl, a normal adult can’t be tied down to one single
	lover.'' The opposite of this fallacy is the Sui Generis Fallacy (also,
	Differance), a postmodern stance that rejects the validity of analogy and
	of inductive reasoning altogether because any given person, place, thing or
	idea under consideration is ``sui generis'' i.e., different and unique, in
	a class unto itself. 

    \subsection{Finish the Job}  The dangerous contemporary fallacy, often aimed
    at a lesser{-}educated or working class audience, that an action or
    standpoint (or the continuation of that action or standpoint) may not be
    questioned or discussed because there is ``a job to be done'' or finished,
    falsely assuming ``jobs'' are meaningless but never to be questioned.
    Sometimes those involved internalize (``buy into'') the ``job'' and make the
    task a part of their own ethos.  (E.g., ``Ours is not to reason why / Ours
    is but to do or die.'') Related to this is the ``Just a Job'' fallacy.
    (E.g., ``How can torturers stand to look at themselves in the mirror? But I
    guess it's OK because for them it's just a job like any other, the job that
    they get paid to do.'')   See also ``Blind Loyalty,'' ``The Soldiers' Honor
    Fallacy'' and the ``Argument from Inertia''.

	\subsection{The Free Speech Fallacy} The infantile fallacy of responding to
	challenges to one's statements and standpoints by whining, ``It's a free
	country, isn't it?  I can say anything I want to!'' A contemporary case of
	this fallacy is the ``Safe Space,'' or ``Safe Place,'' where it is not
	allowed to refute, challenge or even discuss another's beliefs because that
	might be too uncomfortable or ``triggery'' for emotionally fragile
	individuals. E.g., ``All I told him was, 'Jesus loves the little children,'
	but then he turned around and asked me 'But what about birth defects?'
	That's mean. I think I'm going to cry!''  \href{http://billhd.com/}{Prof.
	Bill Hart Davidson} (2017) notes that ``Ironically, the most strident calls
	for 'safety' come from those who want us to issue protections for
	discredited ideas. Things that science doesn't support AND that have
	destroyed lives {-} things like the inherent superiority of one race over
	another. Those ideas wither under demands for evidence. They *are*
	unwelcome. But let's be clear they are unwelcome because they have not
	survived the challenge of scrutiny.'' Ironically, in contemporary America
	``free speech'' has often become shorthand for freedom of racist, offensive
	or even neo{-}Nazi expression, ideological trends that once in power
	typically quash free speech.  Additionally, a recent (2017) scientific
	study has found that, in fact,
	\href{https://www.eurekalert.org/pub_releases/2017-09/bu-pth092717.php}{``people
	think harder and produce better political arguments when their views are
	challenged''} and not artificially protected without challenge.

	\subsection{The Fundamental Attribution Error} (also, Self Justification) A
	corrupt argument from ethos, this fallacy occurs as a result of observing
	and comparing behavior. ``You assume that the bad behavior of others is
	caused by character flaws and foul dispositions while your behavior is
	explained by the environment.  So, for example, I get up in the morning at
	10 a.m.  I say it is because my neighbors party until 2 in the morning
	(situation) but I say that the reason why you do it is that you are lazy.
	Interestingly, it is more common in individualistic societies where we
	value self volition. Collectivist societies tend to look at the environment
	more.  (It happens there, too, but it is much less common.)''  [Thanks to
	scholar Joel Sax for this!]  The obverse of this fallacy is Self
	Deprecation (also Self Debasement), where, out of  either a false humility
	or a genuine lack of self{-}esteem, one deliberately puts oneself down,
	most often in hopes of attracting denials, gratifying compliments and
	praise.

    \subsection{Gaslighting} A recently{-}prominent, vicious fallacy of logic,
    denying or invalidating a person's own knowledge and experiences by
    deliberately twisting or distorting known facts, memories, scenes, events
    and evidence in order to disorient a vulnerable opponent and to make him or
    her doubt his/her sanity. E.g., ``Who are you going to believe?  Me, or
    your own eyes?'' Or, ``You claim you found me in bed with her? Think again!
    You're crazy! You seriously need to see a shrink.'' A very common, though
    cruel instance of Gaslighting that seems to have been particularly familiar
    among mid{-}20th century generations is the fallacy of Emotional
    Invalidation, questioning, after the fact, the reality or ``validity'' of
    affective states, either another's or one's own. E.g., ``Sure, I made it
    happen from beginning to end, but but it wasn't me doing it, it was just my
    stupid hormones betraying me.'' Or, ``You didn't really mean it when you
    said you `hate' Mommy. Now take a time{-}out and you'll feel better.'' Or,
    ``No, you're not really in love; it's just infatuation or `puppy love.' ''
    The fallacy of ``Gaslighting'' is named after British playwright Patrick
    Hamilton's 1938 stage play ``Gas Light,'' also known as ``Angel Street.''
    See also, Blind Loyalty, ``The Big Brain/Little Brain Fallacy,'' The
    Affective Fallacy, and ``Alternative Truth''.

    \subsection{Guilt by Association} The fallacy of trying to refute or condemn
    someone's standpoint, arguments or actions by evoking the negative ethos of
    those with whom the speaker is identified or of a group, party, religion or
    race to which he or she belongs or was once associated with. A form of Ad
    Hominem Argument, e.g., ``Don't listen to her. She's a Republican so you
    can't trust anything she says,'' or ``Are you or have you ever been a
    member of the Communist Party?''  An extreme instance of this is the
    Machiavellian ``For my enemies, nothing'' Fallacy, where real or perceived
    ``enemies'' are by definition always wrong and must be conceded nothing, not
    even the time of day, e.g., ``He's a Republican, so even if he said the sky
    is blue I wouldn't believe him''.

	\subsection{The Half Truth} (also Card Stacking, Stacking the Deck,
	Incomplete Information) A corrupt argument from logos, the fallacy of
	consciously selecting, collecting and sharing only that evidence that
	supports one's own standpoint, telling the strict truth but deliberately
	minimizing or omitting important key details in order to falsify the larger
	picture and support a false conclusion.(E.g. “The truth is that Bangladesh
	is one of the world's fastest{-}growing countries and can boast of a young,
	ambitious and hard{-}working population, a family{-}positive culture, a
	delightful, warm climate of tropical beaches and swaying palms where it
	never snows, low cost medical and dental care, a vibrant faith tradition
	and a multitude of places of worship, an exquisite, world{-}class spicy
	local curry cuisine and a swinging entertainment scene. Taken together, all
	these solid facts clearly prove that Bangladesh is one of the world’s most
	desirable places for young families to live, work and raise a family.”) See
	also, Confirmation Bias.

	\subsection{Hero{-}Busting} (also, ``The Perfect is the Enemy of the
	Good'') A postmodern fallacy of ethos under which, since nothing and nobody
	in this world is perfect there are not and have never been any heroes
	Washington and Jefferson held slaves, Lincoln was (by our contemporary
	standards) a racist, Karl Marx sexually exploited his family's own young
	live{-}in domestic worker and got her pregnant, Martin Luther King Jr. had
	an eye for women too, Lenin condemned feminism, the Mahatma drank his own
	urine (ugh!), Pope Francis is wrong on abortion, capitalism, same{-}sex
	marriage and women's ordination, Mother Teresa loved suffering and was
	wrong on just about everything else too, etc., etc  Also applies to the now
	near{-}universal political tactic of ransacking everything an opponent has
	said, written or done since infancy in order to find something to
	misinterpret or condemn (and we all have something!). An early example of
	this latter tactic is deftly described in Robert Penn Warren's classic
	(1946) novel,
	\href{https://www.amazon.com/All-Kings-Robert-Penn-Warren/dp/0156012952/}{All
	the King's Men}. This is the opposite of the ``Heroes All'' fallacy, below.
	The ``Hero Busting'' fallacy has also been selectively employed at the
	service of the Identity Fallacy (see below) to falsely ``prove'' that ``you
	cannot trust anyone'' but a member of ``our'' identity{-}group since
	everyone else, even the so{-}called ``heroes'' or ``allies'' of other
	groups, are all racist, sexist, anti{-}Semitic, or hate ``us.''  E.g., In
	1862 Abraham Lincoln said he was willing to settle the U.S. Civil War
	either with or without freeing the slaves if it would preserve the Union,
	thus ``conclusively proving'' that all whites are viciously racist at heart
	and that African Americans must do for self and never trust any of
	``them,'' not even those who claim to be allies.

	\subsection{Heroes All} (also, ``Everybody's a Winner'') The contemporary
	fallacy that everyone is above average or extraordinary. A corrupted
	argument from pathos (not wanting anyone to lose or to feel bad). Thus,
	every member of the Armed Services, past or present, who serves honorably
	is a national hero, every student who competes in the Science Fair wins a
	ribbon or trophy, and every racer is awarded a winner's yellow jersey. This
	corruption of the argument from pathos, much ridiculed by disgraced
	American humorist Garrison Keeler, ignores the fact that if everybody wins
	nobody wins, and if everyone's a hero no one's a hero. The logical result
	of this fallacy is that, as children's author Alice Childress writes
	(1973),
	\href{https://www.amazon.com/Hero-Aint-Nothin-but-Sandwich/dp/0698118545/}{``A
	hero ain't nothing but a sandwich.''} See also the ``Soldiers' Honor
	Fallacy.''  

	\subsection{Hoyle's Fallacy} A fallacy of logos, falsely assuming that a
	possible event of low (even vanishingly low) probability can never have
	happened and/or would never happen in real life. E.g., ``The probability of
	something as complex as human DNA emerging by purely random evolution in
	the time the earth has existed is so negligible that it is for all
	practical purposes impossible and must have required divine intervention.''
	Or, ``The chance of a casual, Saturday{-}night poker player being dealt
	four aces off an honest, shuffled deck is so infinitesimal that it would
	never occur even once in a normal lifetime!  That proves you cheated!'' See
	also, Argument from Incredulity. An obverse of Hoyle's Fallacy is ``You
	Can't Win if You Don't Play,'' (also, ``Someone's gonna win and it might as
	well be YOU!'') a common and cruel contemporary fallacy used to persuade
	vulnerable audiences, particularly the poor, the mathematically illiterate
	and gambling addicts to throw their money away on lotteries, horse races,
	casinos and other long{-}shot gambling schemes.

	\subsection{I Wish I Had a Magic Wand} The fallacy of regretfully (and
	falsely) proclaiming oneself powerless to change a bad or objectionable
	situation over which one has power. E.g., ``What can we do about gas
	prices? As Secretary of Energy I wish I had a magic wand, but I don't''
	[shrug] . Or, ``No, you can't quit piano lessons. I wish I had a magic wand
	and could teach you piano overnight, but I don't, so like it or not, you
	have to keep on practicing.'' The parent, of course, ignores the
	possibility that the child may not want or need to learn piano. See also,
	TINA.

	\subsection{The Identity Fallacy} (also Identity Politics; ``Die away, ye
	old forms and logic!'') A corrupt postmodern argument from ethos, a variant
	on the Argumentum ad Hominem in which the validity of one's logic,
	evidence, experience or arguments depends not on their own strength but
	rather on whether the one arguing is a member of a given social class,
	generation, nationality, religious or ethnic group, color, gender or sexual
	orientation, profession, occupation or subgroup. In this fallacy, valid
	opposing evidence and arguments are brushed aside or ``othered'' without
	comment or consideration, as simply not worth arguing about solely because
	of the lack of proper background or ethos of the person making the
	argument, or because the one arguing does not self{-}identify as a member
	of the ``in{-}group.'' E.g., ``You'd understand me right away if you were
	Burmese but since you're not there's no way I can explain it to you,'' or
	``Nobody but another nurse can know what a nurse has to go through.''
	Identity fallacies are reinforced by
	\href{https://www.eurekalert.org/pub_releases/2017-05/afps-grc050517.php}{common
	ritual}, language, and discourse. However, these fallacies are occasionally
	self{-}interested, driven by the egotistical ambitions of academics,
	politicians and would{-}be group leaders anxious to build their own careers
	by carving out a special identity group constituency to the exclusion  of
	existing broader{-}based identities and leadership. An Identity Fallacy may
	lead to scorn or rejection of potentially useful allies, real or
	prospective, because they are not of one's own identity. The Identity
	Fallacy promotes an exclusivist, sometimes cultish ``do for self''
	philosophy which in today's world virtually guarantees
	self{-}marginalization and ultimate defeat.  A recent application of the
	Identity Fallacy is the fallacious accusation of ``Cultural
	Appropriation,'' in which those who are not of the right Identity are
	condemned for ``appropriating'' the cuisine, clothing, language or music of
	a marginalized group, forgetting the old axiom that ``Imitation is the
	sincerest form of flattery.'' Accusations of Cultural Appropriation very
	often stem from competing selfish economic interests (E.g., ``What right do
	those p*nche Gringos have to set up a  taco place right here on Guadalupe
	Drive to take away business from Doña Teresa's Taquería? They even dare to
	play Mexican music in their dining room! That's cultural appropriation!'').
	See also, Othering.

	\subsection{Infotainment} (also Infortainment; Fake News; InfoWars);  A
	very corrupt and dangerous modern media{-}driven fallacy that deliberately
	and knowingly stirs in facts, news, falsities and outright lies with
	entertainment, a mixture usually concocted for specific, base ideological
	and profit{-}making motives. Origins of this fallacy predate the current
	era in the form of ``Yellow'' or ``Tabloid'' Journalism. This deadly
	fallacy has caused endless social unrest, discontent and even shooting wars
	(e.g., the Spanish American War) over the course of modern history.
	Practitioners of this fallacy sometimes hypocritically justify its use on
	the basis that their readers/listeners/viewers ``know beforehand'' (or
	should know) that the content offered is not intended as real news and is
	offered for entertainment purposes only, but in fact this caveat is rarely
	observed by uncritical audiences who eagerly swallow what the purveyors put
	forth. See also Dog{-}Whistle Politics.

    \subsection{The Job's Comforter Fallacy} (also, ``Karma is a bi**h;''  ``What
    goes around comes around.'') The fallacy that since there is no such thing
    as random chance and we (I, my group, or my country) are under special
    protection of heaven, any misfortune or natural disaster that we suffer
    must be a punishment for our own or someone else's secret sin or open
    wickedness. The opposite of the Appeal to Heaven, this is the fallacy
    employed by the Westboro Baptist Church members who protest fallen service
    members' funerals all around the United States. See also, Magical Thinking. 

	\subsection{Just Do it.}  (also, ``Find a way;'' ``I don't care how you do
	it;'' ``Accomplish the mission;'' ``By Any Means Necessary.'' )  A pure,
	abusive Argumentum ad Baculum (argument from force), in which someone in
	power arbitrarily waves aside or overrules the moral objections of
	subordinates or followers and orders them to accomplish a goal by any means
	required, fair or foul  The clear implication is that unethical or immoral
	methods should be used. E.g., ``You say there's no way you can finish the
	dig on schedule because you found an old pioneer gravesite with a fancy
	tombstone on the excavation site? Well, find a way! Make it disappear! Just
	do it! I don't want to know how you do it, just do it! This is a million
	dollar contract and we need  it done by Tuesday.''  See also, Plausible
	Deniability.

	\subsection{Just Plain Folks} (also, ``Values'') This corrupt modern
	argument from ethos argues to a less{-}educated or rural audience that the
	one arguing is ``just plain folks'' who is a ``plain talker,''  ``says what
	s/he is thinking,'' ``scorns political correctness,'' someone who ``you
	don't need a dictionary to understand'' and who thinks like the audience
	and is thus worthy of belief, unlike some member of the fancy{-}talking,
	latte{-}sipping Left Coast Political Elite, some ``double{-}domed
	professor,'' ``inside{-}the{-}beltway Washington bureaucrat,''
	``tree{-}hugger'' or other despised outsider who ``doesn't think like we
	do'' or ``doesn't share our values.''  This is a counterpart to the Ad
	Hominem Fallacy and most often carries a distinct reek of xenophobia or
	racism as well. See also the Plain Truth Fallacy and the Simpleton's
	Fallacy.

	\subsection{The Law of Unintended Consequences} (also, ``Every Revolution
	Ends up Eating its own Young'' Grit; Resilience Doctrine) In this very
	dangerous, archly pessimistic postmodern fallacy the bogus ``Law of
	Unintended Consequences,'' once a semi{-}humorous satirical corollary of
	``Murphy's Law,'' is elevated to to the status of an iron law of history.
	This fallacy arbitrarily proclaims a priori that since we can never know
	everything or securely foresee anything, sooner or later in today's
	``complex world'' unforeseeable adverse consequences and negative side
	effects (so{-}called ``unknown unknowns'') will always end up blindsiding
	and overwhelming, defeating and vitiating any and all naive ``do{-}gooder''
	efforts to improve our world. Instead, one must always expect defeat and be
	ready to roll with the punches by developing ``grit'' or ``resilience'' as
	a primary survival skill. This nihilist fallacy is a practical negation of
	the the possibility of any valid argument from logos. See also, TINA. 

	\subsection{Lying with Statistics} The contemporary fallacy of misusing
	true figures and numbers to “prove” unrelated claims. (e.g. ``In real
	terms, attending college has never been cheaper than it is now. When
	expressed as a percentage of the national debt, the cost of getting a
	college education is actually far less today than it was back in 1965!'').
	A corrupted argument from logos, often preying on the public's perceived or
	actual mathematical ignorance. This includes the Tiny Percentage Fallacy,
	that an amount or action that is quite significant in and of itself somehow
	becomes insignificant simply because it's a tiny percentage of something
	much larger.  E.g., the arbitrary arrest, detention or interception of
	``only'' a few hundred legally{-}boarded international travelers as a tiny
	percentage of the tens of thousands who normally arrive. Under this same
	fallacy a consumer who would choke on spending an extra dollar for two cans
	of peas will typically ignore \$50 extra on the price of a car or \$1000
	extra on the price of a house simply because these differences are ``only''
	a tiny percentage of the much larger amount being spent.  Historically,
	sales taxes or value{-}added taxes (VAT) have successfully gained public
	acceptance and remain ``under the radar'' because of this latter fallacy,
	even though amounting to hundreds or thousands of dollars a year in extra
	tax burden. See also Half{-}truth, the Snow Job, and the Red Herring.

	\subsection{Magical Thinking} (also, the Sin of Presumption; Expect a
	Miracle!) An ancient but deluded fallacy of logos, arguing that when it
	comes to ``crunch time,'' provided one has enough faith, prays hard enough,
	says the right words, does the right rituals, ``names it and claims it,''
	or ``claims the Promise,'' God will always suspend the laws of the universe
	and work a miracle at the request of or for the benefit of the True
	Believer. In practice this nihilist fallacy denies the existence of a
	rational or predictable universe and thus the possibility of any valid
	argument from logic. See also, Positive Thinking, the Appeal to Heaven, and
	the Job's Comforter fallacy.

	\subsection{Mala Fides} (Arguing in Bad Faith; also Sophism)  Using an
	argument that the arguer himself or herself knows is not valid.  E.g., An
	unbeliever attacking believers by throwing verses from their own Holy
	Scriptures at them, or a lawyer arguing for the innocence of someone whom
	s/he knows full well to be guilty. This latter is a common practice in
	American jurisprudence, and is sometimes portrayed as the worst face of
	``Sophism.''  [Special thanks to
	\href{http://www.firstscientist.net/customaac6.html?pid=842560}{Bradley
	Steffens} for pointing out this fallacy!] Included under this fallacy is
	the fallacy of  Motivational Truth (also, Demagogy, or Campaign Promises),
	deliberately lying to ``the people'' to gain their support or motivate them
	toward some action the rhetor perceives to be desirable (using evil
	discursive means toward a ``good'' material end). A particularly bizarre
	and corrupt form of this latter fallacy is Self Deception (also, Whistling
	by the Graveyard). in which one deliberately and knowingly deludes oneself
	in order to achieve a goal, or perhaps simply to suppress anxiety and
	maintain one's energy level, enthusiasm, morale, peace of mind or sanity in
	moments of adversity.

    \subsection{Measurability} A corrupt argument from logos and ethos (that of
    science and mathematics), the modern Fallacy of Measurability proposes that
    if something cannot be measured, quantified and replicated it does not
    exist, or is ``nothing but anecdotal, touchy{-}feely stuff'' unworthy of
    serious consideration, i.e., mere gossip or subjective opinion. Often,
    achieving ``Measurability'' necessarily demands preselecting, ``fiddling'' or
    ``massaging'' the available data simply in order to make it statistically
    tractable, or in order to support a desired conclusion. Scholar Thomas
    Persing thus describes ``The modernist fallacy of falsely and
    inappropriately applying norms, standardizations, and data point
    requirements to quantify productivity or success. This is similar to
    complex question, measurability, and oversimplification fallacies where the
    user attempts to categorize complicated / diverse topics into terms that
    when measured, suit their position. For example, the calculation of
    inflation in the United States doesn't include the changes in the price to
    gasoline, because the price of gasoline is too volatile, despite the fact
    gasoline is necessary for most people to live their lives in the United
    States.'' See also, ``A Priori Argument,'' ``Lying with Statistics,'' and
    the ``Procrustean Fallacy''.

	\subsection{Mind{-}reading} (Also, ``The Fallacy of Speculation;'' ``I can
	read you like a book'') An ancient fallacy, a corruption of stasis theory,
	speculating about someone else's thoughts, emotions, motivations and ``body
	language'' and then claiming to understand these clearly, sometimes more
	accurately than the person in question knows themselves. The rhetor deploys
	this phony ``knowledge'' as a fallacious warrant for or against a given
	standpoint. Scholar Myron Peto offers as an example the baseless claim that
	“Obama doesn’t a da** [sic] for human rights.” Assertions that ``call for
	speculation'' are rightly rejected as fallacious in U.S. judicial
	proceedings but far too often pass uncontested in public discourse. The
	opposite of this fallacy is the postmodern fallacy of Mind Blindness (also,
	the Autist's Fallacy), a complete denial of any normal human capacity for
	``Theory of Mind,'' postulating the utter incommensurability and privacy of
	minds and thus the impossibility of ever knowing or truly understanding
	another's thoughts, emotions, motivations or intents. This fallacy, much
	promoted by the late postmodernist guru Jacques Derrida, necessarily
	vitiates any form of Stasis Theory. However, the Fallacy of Mind Blindness
	has been decisively refuted in several studies, including
	\href{https://www.eurekalert.org/pub_releases/2017-02/afps-wre022217.php}{recent
	(2017) research published by the Association for Psychological Science},
	and a (2017) Derxel University study indicating how
	\href{https://www.eurekalert.org/pub_releases/2017-02/du-bih022417.php}{``our
	minds align when we communicate''}.

	\subsection{Moral Licensing} The contemporary ethical fallacy that one's
	consistently moral life, good behavior or recent extreme suffering or
	sacrifice earns him/her the right to commit an immoral act without
	repercussions, consequences or punishment. E.g., ``I've been good all year,
	so one bad won't matter,'' or  ``After what I've been through, God knows I
	need this.''  The fallacy of Moral Licensing is also sometimes applied to
	nations, e.g., ``Those who criticize repression and the Gulag in the former
	USSR forget what extraordinary suffering the Russians went through in World
	War II and the millions upon millions who died.''  See also Argument from
	Motives.  The opposite of this fallacy is the (excessively rare in our
	times) ethical fallacy of Scruples, in which one obsesses to pathological
	excess about one's accidental, forgotten, unconfessed or unforgiven sins
	and because of them, the seemingly inevitable prospect of eternal
	damnation.

	\subsection{Moral Superiority} (also, Self Righteousness; the Moral High
	Ground)  An ancient, immoral and extremely dangerous fallacy, enunciated in
	Thomistic / Scholastic philosophy in the late Middle Ages, arguing that
	Evil has no rights that the Good and the Righteous are bound to respect.
	That way lies torture, heretic{-}burning, and the Spanish Inquisition.
	Those who practice this vicious fallacy reject any ``moral equivalency''
	(i.e., even{-}handed treatment) between themselves (the Righteous) and
	their enemies (the Wicked), against whom anything is fair, and to whom
	nothing must be conceded, not even the right to life. This fallacy is a
	specific denial of the ancient ``Golden Rule,'' and has been the cause of
	endless intractable conflict, since if one is Righteous no negotiation with
	Evil and its minions is possible; The only imaginable road to a ``just''
	peace is through total victory, i.e., the absolute defeat and liquidation
	of one's Wicked enemies.  American folk singer and Nobel Laureate Bob Dylan
	expertly demolishes this fallacy in his 1963 protest song,
	\href{http://bobdylan.com/songs/god-our-side/}{``With God on Our Side.''}
	See also the Appeal to Heaven, and Moving the Goalposts.

	\subsection{Mortification} (also, Live as Though You're Dying;
	Pleasure{-}hating; No Pain No Gain) An ancient fallacy of logos, trying to
	``beat the flesh into submission'' by extreme exercise or ascetic
	practices, deliberate starvation or infliction of pain, denying the
	undeniable fact that discomfort and pain exist for the purpose of warning
	of lasting damage to the body. Extreme examples of this fallacy are various
	forms of self{-}flagellation such as practiced by the New Mexico
	``Penitentes'' during Holy Week or by Shia devotees during Muharram. More
	familiar contemporary manifestations of this fallacy are extreme
	``insanity'' exercise regimes not intended for normal health, fitness or
	competitive purposes but just to ``toughen'' or ``punish'' the body.
	Certain pop{-}nutritional theories and diets seem based on this fallacy as
	well.  Some contemporary experts suggest that self{-}mortification (an
	English word related to the Latinate French root ``mort,'' or ``death.'')
	is in fact ``suicide on the installment plan.'' Others suggest that it
	involves a narcotic{-}like addiction to the body's natural endorphins. The
	opposite of this fallacy is the ancient fallacy of Hedonism, seeking and
	valuing physical pleasure as a good in itself, simply for its own sake.

	\subsection{Moving the Goalposts} (also, Changing the Rules; All's Fair in
	Love and War; The Nuclear Option; ``Winning isn't everything, it's the only
	thing'') A fallacy of logos, demanding certain proof or evidence, a certain
	degree of support or a certain number of votes to decide an issue, and then
	when this is offered, demanding even more, different or better support in
	order to deny victory to an opponent. For those who practice the fallacy of
	Moral Superiority (above), Moving the Goalposts is often perceived as
	perfectly good and permissible if necessary to prevent the victory of
	Wickedness and ensure the triumph of one's own side, i.e, the Righteous.

	\subsection{MYOB} (Mind Your Own Business;  also You're Not the Boss of Me;
	``None of yer beeswax,'' ``So What?'', The Appeal to Privacy) The
	contemporary fallacy of arbitrarily prohibiting or terminating any
	discussion of one's own standpoints or behavior, no matter how absurd,
	dangerous, evil or offensive, by drawing a phony curtain of privacy around
	oneself and one's actions. A corrupt argument from ethos (one's own). E.g.,
	``Sure, I was doing eighty and weaving between lanes on Mesa
	Street{-}{-}what's it to you? You're not a cop, you're not my nanny. It's
	my business if I want to speed, and your business to get the hell out of my
	way. Mind your own damn business!'' Or, ``Yeah, I killed my baby. So what?
	Butt out! It wasn't your brat, so it's none of your damn business!''
	Rational discussion is cut off because ``it is none of your business!'' See
	also, ``Taboo.'' The counterpart of this is ``Nobody Will Ever Know,''
	(also ``What happens in Vegas stays in Vegas;'' ``I Think We're Alone
	Now,'' or the Heart of Darkness Syndrome) the fallacy that just because
	nobody important is looking (or because one is on vacation, or away in
	college, or overseas) one may freely commit immoral, selfish, negative or
	evil acts at will without expecting any of the normal consequences or
	punishment . Author Joseph Conrad graphically describes this sort of moral
	degradation in the character of Kurtz in his classic novel,
	\href{https://www.amazon.com/Heart-Darkness-Joseph-Conrad/dp/1503275922}{Heart
	of Darkness}.

    \subsection{Name{-}Calling} A variety of the ``Ad Hominem'' argument. The
    dangerous fallacy that, simply because of who one is or is alleged to be,
    any and all arguments, disagreements or objections against one's standpoint
    or actions are automatically racist, sexist, anti{-}Semitic, bigoted,
    discriminatory or hateful. E.g., ``My stand on abortion is the only correct
    one. To disagree with me, argue with me or question my judgment in any way
    would only show what a pig you really are.'' Also applies to refuting an
    argument by simply calling it a ``fallacy,'' or declaring it invalid
    without proving why it is invalid, or summarily dismissing  arguments or
    opponents by labeling them ``racist,'' ``communist,'' ``fascist,''
    ``moron,'' any name followed by the suffix ``tard'' (short for the highly
    offensive ``retard'') or some other negative name without further
    explanation. E.g., ``He's an a**hole, end of story'' or ``I'm a loser.''  A
    subset of this is the Newspeak fallacy, creating identification with a
    certain kind of audience by inventing or using racist or offensive,
    sometimes military{-}sounding nicknames for opponents or enemies, e.g.,
    ``The damned DINO's are even worse than the Repugs and the Neocons.'' Or,
    ``In the Big One it took us only five years to beat both the J*ps and the
    Jerries, so more than a decade and a half after niner{-}eleven why is it so
    hard for us to beat a raggedy bunch of Hajjis and Towel{-}heads?'' Note
    that originally the word ``Nazi'' belonged in this category, but this term
    has long come into use as a proper English noun. See also,
    ``Reductionism,'' ``Ad Hominem Argument,'' and ``Alphabet Soup''.

	\subsection{The Narrative Fallacy} (also, the Fable; the Poster Child) The
	ancient fallacy of persuasion by telling a ``heartwarming'' or horrifying
	story or fable, particularly to less{-}educated or uncritical audiences who
	are less likely to grasp purely logical arguments or general principles.
	E.g., Charles Dickens' ``A Christmas Carol.'' Narratives and fables,
	particularly those that name names and personalize arguments, tend to be
	far more persuasive at a popular level than other forms of argument and are
	virtually irrefutable, even when the story in question is well known to be
	entirely fictional. This fallacy is found even in the field of science, as
	noted by
	\href{https://www.eurekalert.org/pub_releases/2016-12/uow-wmi121616.php}{a
	recent (2017) scientific study}.

    \subsection{The NIMBY Fallacy} (Not in My Back Yard; also ``Build a Wall!'';
    ``Lock'em up and throw away the key;'' The Ostrich Strategy; The Gitmo
    Solution.). The infantile fallacy that a problem, challenge or threat that
    is not physically nearby or to which I am not directly exposed has for all
    practical purposes ``gone away'' and ceased to exist. Thus, a problem can be
    permanently and definitively solved by ``making it go away,'' preferably to
    someplace ``out of sight,'' a walled{-}off  ghetto or a distant isle where
    there is no news coverage, and where nobody important stays. Lacking that,
    it can be made to go away by simply eliminating, censoring or ignoring
    ``negative'' media coverage and public discussion of the problem and
    focusing on ``positive, encouraging'' things instead.

	\subsection{No Discussion} (also No Negotiation; the Control Voice; Peace
	through Strength; a Muscular Foreign Policy; Fascism)  A pure Argumentum ad
	Baculum that rejects reasoned dialogue, offering either instant,
	unconditional compliance/surrender or defeat/death as the only two options
	for settling even minor differences, e.g., screaming ``Get down on the
	ground, now!'' or declaring ``We don't talk to terrorists.'' This deadly
	fallacy falsely paints real or potential ``hostiles'' as monsters devoid of
	all reason, and far too often contains a very strong element of
	``machismo'' as well. I.e. ``A real, muscular leader never resorts to
	pantywaist pleading, apologies, excuses, fancy talk or argument. That's for
	lawyers, liars and pansies and is nothing but a delaying tactic. A real man
	stands tall, says what he thinks, draws fast and shoots to kill.''  The
	late actor John Wayne frequently portrayed this fallacy in his movie roles.
	See also, The Pout.

	\subsection{Non{-}recognition} A deluded fallacy in which one deliberately
	chooses not to publicly ``recognize''  ground truth, usually on the theory
	that this would somehow reward evil{-}doers if we recognize their deeds as
	real or consequential. Often the underlying theory is that the situation is
	``temporary'' and will soon be reversed. E.g., In the decades from 1949
	until Richard Nixon's presidency the United States officially refused to
	recognize the existence of the most populous nation on earth, the People's
	Republic of China, because America supported the U.S.{-}friendly Republic
	of China government on Taiwan instead and hoped they might somehow return
	to power on the mainland. Perversely, in 2016 the U.S. President{-}Elect
	caused a significant international flap by chatting with the President of
	the government on Taiwan, a de facto violation of long{-}standing American
	non{-}recognition of that same regime. More than half a century after the
	Korean War the U.S. still refuses to pronounce the name of, or recognize
	(much less conduct normal, peaceful negotiations with) a nuclear{-}armed
	DPRK (North Korea). An individual who practices this fallacy risks
	institutionalization (e.g., ``I refuse to recognize Mom's murder, 'cuz
	that'd give the victory to the murderer! I refuse to watch you bury her!
	Stop!  Stop!'') but tragically, such behavior is only too common in
	international relations. See also the State Actor Fallacy, Political
	Correctness, and The Pout.

	\subsection{The Non Sequitur} The deluded fallacy of offering evidence,
	reasons or conclusions that have no logical connection to the argument at
	hand (e.g. “The reason I flunked your course is because the U. S.
	government is now putting out purple five{-}dollar bills! Purple!”). (See
	also Red Herring.) Occasionally involves the breathtaking arrogance of
	claiming to have special knowledge of why God, fate, karma or the Universe
	is doing certain things. E.g., ``This week's earthquake was obviously meant
	to punish those people for their great wickedness.'' See also, Magical
	Thinking, and the Appeal to Heaven.

	\subsection{Nothing New Under the Sun} (also, Uniformitarianism, “Seen it
	all before;” ``Surprise, surprise;'' ``Plus ça change, plus c'est la même
	chose.'')  Fairly rare in contemporary discourse, this deeply cynical
	fallacy, a corruption of the argument from logos, falsely proposes that
	there is not and will never be any real novelty in this world. Any argument
	that there are truly “new” ideas or phenomena is judged  a priori to be
	unworthy of serious discussion and dismissed with a jaded sigh and a wave
	of the hand as ``the same old same old.''  E.g., “[Sigh!] Idiots! Don't you
	see that the current influx of refugees from the Mideast is just the same
	old Muslim invasion of Christendom that’s been going on for 1,400 years?”
	Or, “Libertarianism is nothing but re{-}warmed anarchism, which, in turn,
	is nothing but the ancient Antinomian Heresy. Like I told you before,
	there's nothing new under the sun!” 

	\subsection{Olfactory Rhetoric} (also, ``The Nose Knows'') A vicious,
	zoological{-}level fallacy of pathos in which opponents are marginalized,
	dehumanized or hated primarily based on their supposed odor, lack of
	personal cleanliness, imagined diseases or filth. E. g.,  ``Those
	demonstrators are demanding something or another but I'll only talk to them
	if first they go home and take a bath!'' Or, ``I can smell a Jew a block
	away!''  Also applies to demeaning other cultures or nationalities based on
	their differing cuisines, e.g., ``I don't care what they say or do, their
	breath always stinks of garlic. And have you ever smelled their kitchens?''
	Olfactory Rhetoric straddles the borderline between a fallacy and a
	psychopathology.
	\href{https://www.eurekalert.org/pub_releases/2017-08/rb-hte082417.php}{A
	2017 study by Ruhr University Bochum} suggests that olfactory rhetoric does
	not arise from a simple, automatic physiological reaction to an actual
	odor, but in fact, strongly depends on one's predetermined reaction or
	prejudices toward another, and one's olfactory center ``is activated even
	before we perceive an odour.'' See also, Othering. 



    \subsection{Oops!} (also, ``Oh, I forgot...,'' ``The Judicial Surprise,''
    ``The October Surprise,'') A corrupt argument from logos in which toward
    the decisive end of a discussion, debate, trial, electoral campaign period,
    or decision{-}making process an opponent suddenly, elaborately and usually
    sarcastically shams having just remembered or uncovered some salient fact,
    argument or evidence.  E.g., ``Oops, I forgot to ask you  You were
    convicted of this same offense twice before, weren't you?!'' Banned in
    American judicial argument, this fallacy is only too common in public
    discourse. Also applies to supposedly ``discovering'' and sensationally
    reporting some potentially damning information or evidence and then, after
    the damage has been done or the decision has been made, quietly declaring,
    ``Oops, I guess that really wasn't that significant after all. Ignore what
    I said. Sorry 'bout that!'' 

    \subsection{Othering} (also Otherizing, ``They're Not Like Us,'' Stereotyping,
    Xenophobia, Racism, Prejudice) A badly corrupted, discriminatory argument
    from ethos where facts, arguments, experiences or objections are
    arbitrarily disregarded, ignored or put down without serious consideration
    because those involved ``are not like us,'' or ``don't think like us.''
    E.g., ``It's OK for Mexicans to earn a buck an hour in the maquiladoras
    [Mexico{-}based `Twin Plants' run by American or other foreign
    corporations]. If it happened here I'd call it brutal exploitation and
    daylight robbery but south of the border, down Mexico way the economy is
    different and they're not like us.''  Or, ``You claim that life must be
    really terrible over there for terrorists to ever think of blowing
    themselves up with suicide vests just to make a point, but always remember
    that they're different from us. They don't think about life and death the
    same way we do.'' A vicious variety of the Ad Hominem Fallacy, most often
    applied to non{-}white or non{-}Christian populations. A variation on this
    fallacy is the ``Speakee'' Fallacy (``You speakee da English?''; also the
    Shibboleth), in which an opponent's arguments are mocked, ridiculed and
    dismissed solely because of the speaker's alleged or real accent, dialect,
    or lack of fluency in standard English, e.g., ``He told me `Vee vorkers
    need to form a younion!' but I told him I'm not a `vorker,' and to come
    back when he learns to speak proper English.'' A very dangerous, extreme
    example of Othering is Dehumanization, a fallacy of faulty analogy where
    opponents are dismissed as mere cockroaches, lice, apes, monkeys, rats,
    weasels or bloodsucking parasites who have no right to speak or to live at
    all, and probably should be ``squashed like bugs.'' This fallacy is
    ultimately the ``logic'' behind ethnic cleansing, genocide and gas ovens.
    See also the Identity Fallacy, ``Name Calling'' and ``Olfactory Rhetoric.''
    The opposite of this fallacy is the ``Pollyanna Principle'' below.

    \subsection{Overexplanation} A fallacy of logos stemming from the real
    paradox that beyond a certain point, more explanation, instructions, data,
    discussion, evidence or proof inevitably results in less, not more,
    understanding. Contemporary urban mythology holds that this fallacy is
    typically male (``Mansplaining''), while barely half a century ago the
    prevailing myth was that it was men who were naturally monosyllabic,
    grunting or non{-}verbal while women would typically overexplain (e.g., the
    1960 hit song by Joe Jones, ``You Talk Too Much''). ``Mansplaining'' is,
    according to scholar Danelle Pecht, ``the infuriating tendency of many men
    to always have to be the smartest person in the room, regardless of the
    topic of discussion and how much they actually know!''  See also The Snow
    Job, and the ``Plain Truth'' fallacy.

    \subsection{Overgeneralization} (also Hasty Generalization; Totus pro Partes
    Fallacy; the Merological Fallacy) A fallacy of logos where a  broad
    generalization that is agreed to be true is offered as overriding all
    particular cases, particularly special cases requiring immediate attention.
    E.g., ``Doctor, you say that this time of year a  flu vaccination is
    essential. but I would counter that ALL vaccinations are essential''
    (implying that I'm not going to give special attention to getting the flu
    shot). Or, attempting to refute ``Black Lives Matter'' by replying, ``All
    Lives Matter,'' the latter undeniably true but still a fallacious
    overgeneralization in that specific and urgent context.
    Overgeneralization can also mean one sees a single negative outcome as an
    eternal pattern of defeat. Overgeneralization may also include the the Pars
    pro Toto Fallacy, the stupid but common fallacy of incorrectly applying one
    or two true examples to all cases. E.g., a minority person who commits a
    particularly horrifying crime, and whose example is then used to smear the
    reputation of the entire group, or when a government publishes special
    lists of crimes committed by groups who are supposed to be hated, e.g.,
    Jews, or undocumented immigrants. Famously, the case of one Willie Horton
    was successfully used in this manner in the 1988 American presidential
    election to smear African Americans, Liberals, and by extension, Democratic
    presidential candidate Michael Dukakis. See also the fallacy of ``Zero
    Tolerance'' below.

    \subsection{The Paralysis of Analysis} (also, Procrastination; the Nirvana
    Fallacy) A postmodern fallacy that since all data is never in, any
    conclusion is always provisional, no legitimate decision can ever be made
    and any action should always be delayed until forced by circumstances. A
    corruption of the argument from logos. (See also the ``Law of Unintended
    Consequences.'')

	\subsection{The Passive Voice Fallacy} (also, the Bureaucratic Passive) A
	fallacy from ethos, concealing active human agency behind the curtain of
	the grammatical passive voice, e.g., ``It has been decided that you are to
	be let go,'' arrogating an ethos of cosmic infallibility and inevitability
	to a very fallible conscious decision made by identifiable, fallible and
	potentially culpable human beings. Scholar Jackson Katz notes (2017) ``We
	talk about how many women were raped last year, not about how many men
	raped women. We talk about how many girls in a school district were
	harassed last year, not about how many boys harassed girls. We talk about
	how many teenage girls in the state of Vermont got pregnant last year,
	rather than how many men and boys impregnated teenage girls. ...  So you
	can see how the use of the passive voice has a political effect. [It]
	shifts the focus off of men and boys and onto girls and women. Even the
	term `Violence against women' is problematic. It's a passive construction;
	there's no active agent in the sentence. It's a bad thing that happens to
	women, but when you look at the term `violence against women' nobody is
	doing it to them, it just happens to them... Men aren't even a part of
	it.''  See also, Political Correctness. An obverse of the Passive Voice
	Fallacy is the Be{-}verb Fallacy, a cultish linguistic theory and the bane
	of many a first{-}year composition student's life, alleging that an
	extraordinary degree of ``clarity,'' ``sanity,'' or textual ``liveliness''
	can be reached by strictly eliminating all passive verb forms and all forms
	of the verb ``to be'' from English{-}language writing. This odd but
	unproven contention, dating back to Alfred Korzybski's ``General
	Semantics'' self{-}improvement movement of the 1920's and '30's via S. I.
	Hayakawa, blithely ignores the fact that although numerous major world
	languages lack a ubiquitous ``be{-}verb,'' e.g., Russian, Hindi and Arabic,
	speakers of these languages, like English{-}speaking General Semantics
	devotees themselves, have never been proven to enjoy any particular
	cognitive advantage over ordinary everyday users of the passive voice and
	the verb ``to be.'' Nor have writers of the curiously stilted English that
	results from applying this fallacy achieved any special success in
	academia, professional or technical writing, or in the popular domain.

    \subsection{Paternalism} A serious fallacy of ethos, arbitrarily
    tut{-}tutting, dismissing or ignoring another's arguments or concerns as
    ``childish'' or ``immature;'' taking a condescending attitude of superiority
    toward opposing standpoints or toward opponents themselves. E.g., ``Your
    argument against the war is so infantile. Try approaching the issue like an
    adult for a change,'' ``I don't argue with children,'' or ``Somebody has to
    be the grownup in the room, and it might as well be me. Here's why you're
    wrong...''  Also refers to the sexist fallacy of dismissing a woman's
    argument because she is a woman, e.g., ``Oh, it must be that time of the
    month, eh?'' See also ``Ad Hominem Argument'' and ``Tone Policing''.

    \subsection{Personalizaion} A deluded fallacy of ethos, seeing yourself or
    someone else as the essential cause of some external event for which you or
    the other person had no responsibility. E.g., ``Never fails! It had to
    happen! It's my usual rotten luck that the biggest blizzard of the year had
    to occur just on the day of our winter festival. If it wasn't for ME being
    involved I'm sure the blizzard wouldn't have happened!'' This fallacy can
    also be taken in a positive sense, e.g. Hitler evidently believed that
    simply because he was Hitler every bullet would miss him and no explosive
    could touch him. ``Personalization'' straddles the borderline between a
    fallacy and a psychopathology. See also, ``The Job's Comforter Fallacy,''
    and ``Magical Thinking''.

	\subsection{The Plain Truth Fallacy; }(also, the Simple Truth fallacy,
	Salience Bias, the KISS Principle [Keep it Short and Simple / Keep it
	Simple, Stupid], the Monocausal Fallacy; the Executive Summary) A fallacy
	of logos favoring familiar, singular, summarized or easily comprehensible
	data, examples, explanations and evidence over those that are more complex
	and unfamiliar but much closer to the truth. E.g., ``Ooooh, look at all
	those equations and formulas!  Just boil it down to the Simple Truth,'' or
	``I don't want your damned philosophy lesson!  Just tell me the Plain Truth
	about why this is happening.''  A more sophisticated version of this
	fallacy arbitrarily proposes, as did 18th century Scottish rhetorician John
	Campbell, that the Truth is always simple by nature and only malicious
	enemies of  Truth would ever seek to make it complicated. (See also, The
	Snow Job, and Overexplanation.) The opposite of this is the postmodern
	fallacy of Ineffability or Complexity (also, Truthiness; Post{-}Truth),,
	arbitrarily declaring that today's world is so complex that there is no
	truth, or that Truth (capital{-}T), if indeed such a thing exists, is
	unknowable except perhaps by God or the Messiah and is thus forever
	inaccessible and irrelevant to us mere mortals, making any cogent argument
	from logos impossible. See also the Big Lie, and Paralysis of Analysis.

    \subsection{Plausible Deniability}  A vicious fallacy of ethos under which
    someone in power forces those under his or her control to do some
    questionable or evil act and to then falsely assume or conceal
    responsibility for that act in order to protect the ethos of the one in
    command. E.g., ``Arrange a fatal accident but make sure I know nothing
    about it!'' 

	\subsection{Playing on Emotion} (also, the Sob Story; the Pathetic Fallacy;
	the ``Bleeding Heart'' fallacy, the Drama Queen / Drama King Fallacy) The
	classic fallacy of pure argument from pathos, ignoring facts and evoking
	emotion alone. E.g., “If you don’t agree that witchcraft is a major problem
	just shut up, close your eyes for a moment and picture in your mind all
	those poor moms crying bitter tears for their innocent tiny children whose
	cozy little beds and happy tricycles lie all cold and abandoned, just
	because of those wicked old witches! Let's string’em all up!” The opposite
	of this is the Apathetic Fallacy (also, Cynicism; Burnout; Compassion
	Fatigue), where any and all legitimate arguments from pathos are brushed
	aside because, as noted country music artist Jo Dee Messina sang (2005),
	``My give{-}a{-}damn's busted.'' Obverse to Playing on Emotion is the
	ancient fallacy of Refinement (``Real Feelings''), where certain classes of
	living beings such as plants and non{-}domesticated animals, infants,
	babies and minor children, barbarians, slaves, deep{-}sea sailors,
	farmworkers, criminals and convicts, refugees, addicts, terrorists,
	Catholics, Jews, foreigners, the poor, people of color, ``Hillbillies,''
	``Hobos,'' homeless or undocumented people, or ``the lower classes'' in
	general are deemed incapable of experiencing real pain like we do, or of
	having any ``real feelings'' at all, only brutish appetites, vile lusts,
	evil drives, filthy cravings, biological instincts, psychological reflexes
	and automatic tropisms. Noted rhetorician Kenneth Burke falls into this
	last, behaviorist fallacy in his otherwise brilliant (1966)
	\href{https://www.amazon.com/Language-As-Symbolic-Action-Literature/dp/0520001923/}{Language
	as Symbolic Action}, in his discussion of a bird trapped in a lecture room.
	See also, Othering.

	\subsection{Political Correctness (``PC'')} A postmodern fallacy, a
	counterpart of the ``Name Calling'' fallacy, supposing that the nature of a
	thing or situation can be changed by simply changing its name. E.g.,
	``Today we strike a blow for animal rights and against cruelty to animals
	by changing the name of ‘pets’ to ‘animal companions.’'' Or ``Never, ever
	play the `victim' card, because it's so manipulative and sounds so
	negative, helpless and despairing. Instead of being `victims,' we are proud
	to be `survivors.' '' (Of course, when ``victims'' disappear then
	perpetrators conveniently vanish as well!)  See also, The Passive Voice
	Fallacy, and The Scripted Message.  Also applies to other forms of
	political ``Language Control,'' e.g., being careful never to refer to North
	Korea or ISIS/ISIL by their rather pompous proper names (``the Democratic
	People's Republic of Korea'' and ``the Islamic State,'' respectively) or to
	the Syrian government as the ``Syrian government,'' (It's always the
	``Regime'' or the ``Dictatorship.''). Occasionally the fallacy of
	``Political Correctness'' is falsely confused with simple courtesy, e.g.,
	``I'm sick and tired of the tyranny of Political Correctness, having to
	watch my words all the time{-}{-}I want to be free to speak my mind and to
	call out a N{-}{-}{-}{-}{-} or a Queer in public any time I damn well feel
	like it!'' See also, Non{-}recognition. An opposite of this fallacy is the
	fallacy of Venting, below.

	\subsection{The Pollyanna Principle} (also, ``The Projection Bias,''
	``They're Just Like Us,'' ``Singing 'Kumbaya.' '')  A traditional, often
	tragic fallacy of ethos, that of automatically (and falsely) assuming that
	everyone else in any given place, time and circumstance had or has
	basically the same (positive) wishes, desires, interests, concerns, ethics
	and moral code as ``we'' do. This fallacy practically if not theoretically
	denies both the reality of difference and the human capacity to chose
	radical evil.  E.g., arguing that ``The only thing most Nazi Storm Troopers
	wanted was the same thing we do, to live in peace and prosperity and to
	have a good family life,'' when the reality was radically otherwise. Dr.
	William Lorimer offers this explanation ``The Projection Bias is the flip
	side of the `They're Not Like Us' [Othering] fallacy. The Projection bias
	(fallacy) is 'They're just people like me, therefore they must be motivated
	by the same things that motivate me.' For example `I would never pull a gun
	and shoot a police officer unless I was convinced he was trying to murder
	me; therefore, when Joe Smith shot a police officer, he must have been in
	genuine fear for his life.' I see the same fallacy with regard to Israel
	`The people of Gaza just want to be left in peace; therefore, if Israel
	would just lift the blockade and allow Hamas to import anything they want,
	without restriction, they would stop firing rockets at Israel.' That may or
	may not be true {-} I personally don't believe it {-} but the argument
	clearly presumes that the people of Gaza, or at least their leaders, are
	motivated by a desire for peaceful co{-}existence.'' The Pollyanna
	Principle was gently but expertly demolished in the classic
	twentieth{-}century American animated cartoon series, ``The Flintstones,''
	in which the humor lay in the absurdity of picturing ``Stone Age''
	characters having the same concerns, values and lifestyles as
	mid{-}twentieth century white working class Americans.  This is the
	opposite of the Othering fallacy. (Note The Pollyanna Principle fallacy
	should not be confused with a psychological principle of the same name
	which observes that positive memories are usually retained more strongly
	than negative ones. )   

	\subsection{The Positive Thinking Fallacy} An immensely popular but deluded
	modern fallacy of logos, that because we are ``thinking positively'' that
	in itself somehow biases external, objective reality in our favor even
	before we lift a finger to act. See also, Magical Thinking. Note that this
	particular fallacy is often part of a much wider closed{-}minded, somewhat
	cultish ideology where the practitioner is warned against paying attention
	to to or even acknowledging the reality of evil, or of ``negative''
	evidence or counter{-}arguments against his/her standpoints. In the latter
	case rational discussion, argument or refutation is most often futile. See
	also, Deliberate Ignorance.

    \subsection{The Post Hoc Argument} (also, ``Post Hoc Propter Hoc;''  ``Post
    Hoc Ergo Propter Hoc;'' ``Too much of a coincidence,'' the ``Clustering
    Illusion'') The classic paranoiac fallacy of attributing an imaginary
    causality to random coincidences, concluding that just because something
    happens close to, at the same time as, or just after something else, the
    first thing is caused by the second. E.g., ``AIDS first emerged as a
    epidemic back in the very same era when Disco music was becoming
    popular{-}{-}that's too much of a coincidence It proves that Disco caused
    AIDS!''  Correlation does not equal causation.

    \subsection{The Pout} (also The Silent Treatment; Nonviolent Civil
    Disobedience; Noncooperation) An often{-}infantile Argumentum ad Baculum
    that arbitrarily rejects or gives up on dialogue before it is concluded.
    The most benign nonviolent form of this fallacy is found in
    passive{-}aggressive tactics such as slowdowns, boycotts, lockouts,
    sitdowns and strikes.  Under President Barack Obama the United States
    finally ended a half{-}century long political Pout with Cuba. See also ``No
    Discussion'' and ``Nonrecognition''.

	\subsection{The Procrustean Fallacy} (also, ``Keeping up Standards,''
	Standardization, Uniformity, Fordism).  The modernist fallacy of falsely
	and inappropriately applying the norms and requirements of standardized
	manufacturing. quality control and rigid scheduling, or of military
	discipline to inherently diverse free human beings, their lives, education,
	behavior, clothing and appearance. This fallacy often seems to stem from
	the pathological need of someone in power to place in ``order'' their
	disturbingly free, messy and disordered universe by restricting others'
	freedom and insisting on rigid standardization, alphabetization,
	discipline, uniformity and ``objective'' assessment of everyone under their
	power. This fallacy partially explains why marching in straight lines, mass
	calisthenics, goose{-}stepping, drum{-}and{-}bugle or flag corps, standing
	at attention, saluting, uniforms, and standardized categorization are so
	typical of fascism, tyrannical regimes, and of tyrants petty and grand
	everywhere. Thanks to author Eimar O'Duffy for identifying this fallacy!

	\subsection{Prosopology} (also, Prosopography, Reciting the Litany; ``Tell
	Me, What Were Their Names?''; Reading the Roll of Martyrs) An ancient
	fallacy of pathos and ethos, publicly reading out loud, singing, or
	inscribing at length a list of names (most or all of which will be unknown
	to the reader or audience), sometimes in a negative sense, to underline the
	gravity of a past tragedy or mass{-}casualty event, sometimes in a positive
	sense, to emphasize the ancient historical continuity of a church,
	organization or cause. Proper names, especially if they are from the same
	culture or language group as the audience, can have near{-}mystical
	persuasive power.  In some cases, those who use this fallacy in its
	contemporary form will defend it as an attempt to ``personalize'' an
	otherwise anonymous recent mass tragedy. This fallacy was virtually unknown
	in secular American affairs before about 100 years ago, when the custom
	emerged of listing of the names of local World War I casualties on
	community monuments around the country. That this is indeed a fallacy is
	evident by the fact that the names on these century{-}old monuments are now
	meaningful only to genealogists and specialized historians, just as the
	names on the Vietnam War Memorial in Washington or the names of those who
	perished on 9/11 will surely be in another several generations.

	\subsection{The Red Herring} (also, Distraction) An irrelevant argument,
	attempting to mislead and distract an audience by bringing up an unrelated
	but emotionally loaded issue. E.g., ``In regard to my several bankruptcies
	and recent indictment for corruption let’s be straight up about what’s
	really important Terrorism!  Just look at what happened last week in [name
	the place]. Vote for me and I'll fight those terrorists anywhere in the
	world!''  Also applies to raising unrelated issues as falsely opposing the
	issue at hand, e.g., ``You say `Black Lives Matter,' but I would rather say
	`Climate Change Matters!' '' when the two contentions are in no way opposed,
	only competing for attention. See also Availability Bias, and Dog Whistle
	Politics.

	\subsection{Reductio ad Hitlerum} (or, ad Hitleram) A highly problematic
	contemporary historical{-}revisionist contention that the argument ``That's
	just what Hitler said (or would have said, or would have done)'' is a
	fallacy, an instance of the Ad Hominem argument and/or Guilt by
	Association. Whether the Reductio ad Hitlerum can be considered an actual
	fallacy or not seems to fundamentally depend on one's personal view of
	Hitler and the gravity of his crimes.

	\subsection{Reductionism} (also, Oversimplifying, Sloganeering) The fallacy
	of deceiving an audience by giving simple answers or bumper{-}sticker
	slogans in response to complex questions, especially when appealing to less
	educated or unsophisticated audiences. E.g., ``If the glove doesn’t fit,
	you must vote to acquit.'' Or, ``Vote for Snith. He'll bring back jobs!''
	In science, technology, engineering and mathematics (``STEM subjects'')
	reductionism is intentionally practiced to make intractable problems
	computable, e.g., the well{-}known humorous suggestion,
	\href{https://www.wired.com/2011/02/what-is-up-with-the-spherical-cow/}{``First,
	let's assume the cow is a sphere!''}. See also, the Plain Truth Fallacy,
	and Dog{-}whistle Politics.

	\subsection{Reifying} (also, Mistaking the Map for the Territory) The
	ancient fallacy of treating imaginary intellectual categories, schemata or
	names as actual, material ``things.'' (E.g., ``The War against Terror is
	just another chapter in the eternal fight to the death between Freedom and
	Absolute Evil!'') Sometimes also referred to as ``Essentializing'' or
	“Hypostatization.”

    \subsection{The Romantic Rebel} (also, the Truthdig / Truthout Fallacy; the
    Brave Heretic; Conspiracy theories; the Iconoclastic Fallacy) The
    contemporary fallacy of claiming Truth or validity for one's standpoint
    solely or primarily because one is supposedly standing up heroically to the
    dominant ``orthodoxy,'' the current Standard Model, conventional wisdom or
    Political Correctness, or whatever may be the Bandwagon of the moment; a
    corrupt argument from ethos. E.g., ``Back in the day the scientific
    establishment thought that the world was flat, that was until Columbus
    proved them wrong!  Now they want us to believe that ordinary water is
    nothing but H2O. Are you going to believe them? The government is
    frantically trying to suppress the truth that our public drinking{-}water
    supply actually has nitrogen in it and causes congenital vampirism! And
    what about Area 51? Don't you care? Or are you just a kiss{-}up for the
    corrupt scientific establishment?'' The opposite of the Bandwagon fallacy.

	\subsection{The ``Save the Children'' Fallacy} (also, Humanitarian Crisis)
	A cruel and cynical contemporary media{-}driven fallacy of pathos, an
	instance of the fallacious Appeal to Pity, attracting public support for
	intervention in somebody else's crisis in a distant country by repeatedly
	showing in gross detail the extreme (real) suffering of the innocent,
	defenseless little children (occasionally extended even to their pets!) on
	``our'' side, conveniently ignoring the reality that innocent children on
	all sides usually suffer the most in any war, conflict, famine or crisis.
	Recent (2017) examples include the so{-}called ``Rohingya'' in
	Myanmar/Burma (ignoring multiple other ethnicities suffering ongoing hunger
	and conflict in that impoverished country), children in rebel{-}held areas
	of Syria (areas held by our rebels, not by the Syrian government or by
	Islamic State rebels), and the children of Mediterranean boat{-}people
	(light complected children from the Mideast, Afghanistan and North Africa,
	but not darker, African{-}complected children from sub{-}Saharan countries,
	children who are evidently deemed by the media to be far less worthy of
	pity). Scholar Glen Greenwald points out that a cynical key part of this
	tactic is hiding the child and adult victims of one's own violence while
	``milking'' the tragic, blood{-}soaked images of children killed by the
	``other side'' for every tear they can generate as a causus belli [a
	puffed{-}up excuse for war, conflict or American/Western intervention].

	\subsection{Scapegoating} (also, Blamecasting) The ancient fallacy that
	whenever something goes wrong there's always someone other than oneself to
	blame. Although sometimes this fallacy is a practical denial of randomness
	or chance itself, today it is more often a mere insurance{-}driven business
	decision (``I don't care if it was an accident! Somebody with deep pockets
	is gonna pay for this!''), though often scapegoating is no more than a
	cynical ploy to shield those truly responsible from blame. The term
	``Scapegoating'' is also used to refer to the tactic of casting collective
	blame on marginalized or scorned ``Others,'' e.g., ``The Jews are to
	blame!'' A particularly corrupt and cynical example of scapegoating is the
	fallacy of Blaming the Victim, in which one falsely casts the blame for
	one's own evil or questionable actions on those affected, e.g., ``If you
	move an eyelash I'll have to kill you and you'll be to blame!'' ``If you
	don't bow to our demands we'll shut down the government and it'll be
	totally YOUR fault!'' or ``You bi**h, you acted flirty and made me rape
	you! Then you snitched on me to the cops and let them collect a rape kit on
	you, and now I'm going to prison and every bit of it is your fault!'' See
	also, the Affective Fallacy.

	\subsection{Scare Tactics} (also Appeal to Fear; Paranoia; the Bogeyman
	Fallacy; Shock Doctrine [ShockDoc]; Rally `Round the Flag; Rally 'Round the
	President) A variety of Playing on Emotions, a corrupted argument from
	pathos, taking advantage of a emergent or deliberately{-}created crisis and
	its associated public shock, panic and chaos in order to impose an
	argument, action or solution that would be clearly unacceptable if
	carefully considered. E.g., ``If you don't shut up and do what I say we're
	all gonna die! In this moment of crisis we can't afford the luxury of
	criticizing or trying to second{-}guess my decisions when our very lives
	and freedom are in peril!  Instead, we need to be united as one!'' Or, in
	the (2017) words of former White House Spokesperson Sean Spicer, ``This is
	about the safety of America!'' This fallacy is discussed at length in Naomi
	Klein's (2010)
	\href{https://www.amazon.com/Shock-Doctrine-Rise-Disaster-Capitalism/dp/080507983}{The
	Shock Doctrine The Rise of Disaster Capitalism} and her (2017)
	\href{https://www.amazon.com/No-Not-Enough-Resisting-Politics/dp/1608468909}{No
	is Not Enough Resisting Trump's Shock Politics and Winning the World We
	Need}. See also, The Shopping Hungry Fallacy, Dog{-}Whistle Politics, ``We
	Have to do Something!'', and The Worst Case Fallacy.

    \subsection{``Scoring''} (also, Moving the Ball Down the Field, the Sports
    World Fallacy; ``Hey, Sports Fans!'') An instance of faulty analogy, the
    common contemporary fallacy of inappropriately and most often offensively
    applying sports, gaming, hunting or other recreational imagery to unrelated
    areas of life, such as war or intimacy. E.g., ``Nope, I haven't scored with
    Francis yet, but last night I managed to get to third base!''  or ``We
    really need to take our ground game into Kim's half of the field if we ever
    expect to score against North Korea.'' This fallacy is almost always soaked
    in testosterone and machismo. An associated fallacy is that of Evening up
    the Score (also, Getting Even), exacting tit{-}for{-}tat vengeance as
    though life were some sort of ``point{-}score'' sports contest.
    Counter{-}arguments to the ``Scoring'' fallacy usually fall on deaf ears,
    since the one and only purpose for playing a game is to ``score,'' isn't it?

    \subsection{The Scripted Message} (also, Talking Points)  A contemporary
    fallacy related to Big Lie Technique, where a politician or public figure
    strictly limits her/his statements on a given issue to repeating carefully
    scripted, often exaggerated or empty phrases developed to achieve maximum
    acceptance or maximum desired reaction from a target audience. See also,
    Dog Whistle Politics, and Political Correctness, above. The opposite of
    this fallacy is that of ``Venting''.

    \subsection{Sending the Wrong Message} A dangerous fallacy of logos that
    attacks a given statement, argument or action, no matter how good, true or
    necessary, because it will ``send the wrong message.'' In effect, those who
    use this fallacy are openly confessing to fraud and admitting that the
    truth will destroy the fragile web of illusion they have deliberately
    created by their lies. E.g., ``Actually, we haven't a clue about how to
    deal with this crisis, but if we publicly admit it we'll be sending the
    wrong message.'' See also, ``Mala Fides.'' 

    \subsection{Shifting the Burden of Proof}  A classic fallacy of logos that
    challenges an opponent to disprove a claim rather than asking the person
    making the claim to defend his/her own argument. E.g., ``These days
    space{-}aliens are everywhere among us, masquerading as true humans, even
    right here on campus! I dare you to prove it isn't so! See?  You can't! You
    admit it! That means what I say has to be true. Most probably, you're one
    of them, since you seem to be so soft on space{-}aliens!'' A typical tactic
    in using this fallacy is first to get an opponent to admit that a
    far{-}fetched claim, or some fact related to it, is indeed at least
    theoretically ``possible,'' and then declare the claim ``proven'' absent
    evidence to the contrary. E.g., ``So you admit that massive undetected
    voter fraud is indeed possible under our current system, and could have
    happened in this country at least in theory, and you can't produce even the
    tiniest scintilla of evidence that it didn't actually happen! Ha{-}ha! I
    rest my case.'' See also, Argument from Ignorance. 

    \subsection{The Shopping Hungry Fallacy} A fallacy of pathos, a variety of
    Playing on Emotions and sometimes Scare Tactics, making stupid but
    important decisions (or being prompted, manipulated or forced to ``freely''
    take public or private decisions that may be later regretted but are
    difficult to reverse) ``in the heat of the moment'' when  under the
    influence of strong emotion (hunger, fear, lust, anger, sadness, regret,
    fatigue, even joy, love or happiness). E.g., Trevor Noah, (2016) host of
    the Daily Show on American television attributes public approval of
    draconian measures in the Patriot Act and the creation of the U. S.
    Department of Homeland Security to America's ``shopping hungry'' immediately
    after 9/11. See also, Scare Tactics; ``We Have to Do Something;'' and The
    Big ``But'' Fallacy.

    \subsection{The Silent Majority Fallacy} A variety of the argument from
    ignorance, this fallacy, famously enunciated by disgraced American
    President Richard Nixon, alleges special knowledge of a hidden ``silent
    majority'' of voters (or of the population in general) that stands in
    support of an otherwise unpopular leader and his/her policies, contrary to
    the repeated findings of polls, surveys and popular vote totals. In an
    extreme case the leader arrogates to him/herself the title of the ``Voice
    of the Voiceless''.

	\subsection{The Simpleton's Fallacy}  (Or, The ``Good Simpleton'' Fallacy)
	A corrupt fallacy of logos, described in an undated quote from science
	writer Isaac Asimov as ``The false notion that democracy means that `my
	ignorance is just as good as your knowledge.' '' The name of this fallacy
	is borrowed from Walter M. Miller Jr.'s classic (1960) post{-}apocalyptic
	novel, A
	\href{https://www.amazon.com/Canticle-Leibowitz-Walter-Miller-Jr/dp/0060892994/}{Canticle
	for Leibowitz}, in which in the centuries after a nuclear holocaust
	knowledge and learning become so despised that ``Good Simpleton'' becomes
	the standard form of interpersonal salutation. This fallacy is masterfully
	portrayed in the person of the title character in the 1994 Hollywood movie,
	``Forrest Gump.'' The fallacy is widely alleged to have had a great deal to
	do with the outcome of the 2016 US presidential election, See also ``Just
	Plain Folks,'' and the ``Plain Truth Fallacy.'' U.S. President Barrack
	Obama noted to the contrary (2016), ``In politics and in life, ignorance is
	not a virtue. It's not cool to not know what you're talking about. That's
	not real or telling it like it is. That's not challenging political
	correctness. That's just not knowing what you're talking about.'' The term
	``Simpleton's Fallacy'' has also been used to refer to a deceptive
	technique of argumentation, feigning ignorance in order to get one's
	opponent to admit to, explain or overexplain something s/he would rather
	not discuss.  E.g., ``I see here that you have a related prior conviction
	for something called `Criminal Sodomy.' I may be a poor, naive simpleton
	but I'm not quite sure what that fine and fancy lawyer{-}talk means in
	plain English.  Please explain to the jury in simple terms what exactly you
	did to get convicted of that crime.'' See also, Argument from Ignorance,
	and The Third Person Effect.

    \subsection{The Slippery Slope} (also, the Domino Theory) The common fallacy
    that ``one thing inevitably leads to another.'' E.g., ``If you two go and
    drink coffee together one thing will lead to another and next thing you
    know you'll be pregnant and end up spending your life on welfare living in
    the Projects,'' or ``If we close Gitmo one thing will lead to another and
    before you know it armed terrorists will be strolling through our church
    doors with suicide belts, proud as you please, smack in the middle of the
    1030 a.m. Sunday worship service right here in Garfield, Kansas''!

    \subsection{The Snow Job} (also Falacia ad Verbosium; Information Bias) A
    fallacy of logos, “proving” a claim by overwhelming an audience (``snowing
    them under'') with mountains of true but marginally{-}relevant  documents,
    graphs, words, facts, numbers, information and statistics that look
    extremely impressive but which the intended audience cannot be expected to
    understand or properly evaluate. This is a corrupted argument from logos.
    See also, ``Lying with Statistics.'' The opposite of this fallacy is the Plain Truth Fallacy.

	\subsection{The Soldiers' Honor Fallacy} The ancient fallacy that all who
	wore a uniform, fought hard and followed orders are worthy of some special
	honor or glory or are even ``heroes,'' whether they fought for freedom or
	fought to defend slavery, marched under Grant or Lee, Hitler, Stalin,
	Eisenhower or McArthur, fought to defend their homes, fought for oil or to
	spread empire, or even fought against and killed U.S. soldiers! A corrupt
	argument from ethos (that of a soldier), closely related to the ``Finish
	the Job'' fallacy (``Sure, he died for a lie, but he deserves honor because
	he followed orders and did his job faithfully to the end!''). See also
	``Heroes All.'' This fallacy was recognized and decisively refuted at the
	Nuremburg Trials after World War II but remains powerful to this day
	nonetheless. See also ``Blind Loyalty.'' Related is the State Actor
	Fallacy, that those who fight and die for their country (America, Russia,
	Iran, the Third Reich, etc.) are worthy of honor or at least pardonable
	while those who fight for a non{-}state actor (armed abolitionists,
	guerrillas, freedom{-}fighters, jihadis, mujahideen) are not and remain
	``terrorists'' no matter how noble or vile their cause, until or unless
	they win and become the recognized state, or are adopted by a state after
	the fact.

	\subsection{The Standard Version Fallacy}  The ancient fallacy, a
	discursive Argumentum ad Baculum, of choosing a ``Standard Translation'' or
	``Authorized Version'' of an  ancient or sacred text and arbitrarily
	declaring it ``correct'' and ``authoritative,'' necessarily eliminating
	much of the poetry and underlying meaning of the original but conveniently
	quashing any further discussion about the meaning of the original text,
	e.g., the Vulgate or The King James Version. The easily demonstrable fact
	that translation (beyond three or four words) is neither uniform nor
	reversible (i.e., never comes back exactly the same when retranslated from
	another language) gives the lie to any efforts to make translation of human
	languages into an exact science. Islam clearly recognizes this fallacy when
	characterizing any attempt to translate the sacred text of the Holy Qur'an
	out of the original Arabic as a ``paraphrase'' at very best. An obverse of
	this fallacy is the Argumentum ad Mysteriam, above.  An extension of the
	Standard Version Fallacy is the Monolingual Fallacy, at an academic level
	the fallacy of ignorantly assuming (as a monolingual person) that
	transparent, in{-}depth translation between languages is the norm, or even
	possible at all, allowing one to conveniently and falsely ignore everyday
	issues of translation when close{-}reading translated literature or
	academic text and theory. At the popular level the Monolingual Fallacy
	allows monolinguals to blithely demand that visitors, migrants, refugees
	and newcomers learn English, either before arriving or else overnight after
	arrival in the United States, while applying no such demand to themselves
	when they go to Asia, Europe, Latin America, or even French{-}speaking
	areas of Canada. Not rarely, this fallacy descends into gross racism or
	ethnic discrimination, e.g., the demagogy of warning of ``Spanish being
	spoken right here on Main Street and taco trucks on every corner!'' See
	also, Othering, and Dog{-}Whistle Politics.

	\subsection{Star Power} (also Testimonial, Questionable Authority, Faulty
	Use of Authority, Falacia ad Vericundiam; Eminence{-}based Practice) In
	academia and medicine, a corrupt argument from ethos in which arguments,
	standpoints and themes of professional discourse are granted fame and
	validity or condemned to obscurity solely by whoever may be the reigning
	``stars'' or ``premier journals'' of the profession or discipline at the
	moment. E.g., ``Foster's take on Network Theory has been thoroughly
	criticized and is so last{-}week!.This week everyone's into Safe Spaces and
	Pierce's Theory of Microaggressions. Get with the program.'' (See also, the
	Bandwagon.) Also applies to an obsession with journal Impact Factors. At
	the popular level this fallacy also refers to a corrupt argument from ethos
	in which public support for a standpoint or product is established by a
	well{-}known or respected figure (i.e.,. a star athlete or entertainer) who
	is not an expert and who may have been well paid to make the endorsement
	(e.g., ``Olympic gold{-}medal pole{-}vaulter Fulano de Tal uses Quick Flush
	Internet{-}{-}Shouldn’t you?'' Or, ``My favorite rock star warns that
	vaccinations spread cooties, so I'm not vaccinating my kids!'' ). Includes
	other false, meaningless or paid means of associating oneself or one’s
	product or standpoint with the ethos of a famous person or event (e.g.,
	“Try Salsa Cabria, the official taco sauce of the Winter Olympics!”). This
	fallacy also covers Faulty use of Quotes (also, The Devil Quotes
	Scripture), including quoting out of context or against the clear intent of
	the original speaker or author.  E.g., racists quoting and twisting the
	Rev. Dr. Martin Luther King Jr.'s statements in favor of racial equality
	against contemporary activists and movements for racial equality. 

	\subsection{The Straw Man} (also ``The Straw Person'' ``The Straw Figure'')
	The fallacy of setting up a phony, weak, extreme or ridiculous parody of an
	opponent's argument and then proceeding to knock it down or reduce it to
	absurdity with a rhetorical wave of the hand. E.g., ``Vegetarians say
	animals have feelings like you and me. Ever seen a cow laugh at a
	Shakespeare comedy? Vegetarianism is nonsense!'' Or, ``Pro{-}choicers hate
	babies and want to kill them!'' Or, ``Pro{-}lifers hate women and want them
	to spend their lives barefoot, pregnant and chained to the kitchen stove!''
	A too{-}common example of this fallacy is that of highlighting the most
	absurd, offensive, silly or violent examples in a mass movement or
	demonstration, e.g. ``Tree huggers'' for environmentalists, ``bra burners''
	for feminists, or ``rioters'' when there are a dozen violent crazies in a
	peaceful, disciplined demonstration of thousands or tens of thousands, and
	then falsely portraying these extreme examples as typical of the entire
	movement in order to condemn it with a wave of the hand. See also Olfactory
	Rhetoric.

    \subsection{The Taboo} (also, Dogmatism) The ancient fallacy of unilaterally
    declaring certain ``bedrock'' arguments, assumptions, dogmas, standpoints
    or actions ``sacrosanct'' and not open to discussion, or arbitrarily taking
    some emotional tones, logical standpoints, doctrines or options ``off the
    table'' beforehand. (E.g., ``No, let's not discuss my sexuality,''
    ``Don't bring my drinking into this,'' or ``Before we start, you need to
    know I won't allow you to play the race card or permit you to attack my
    arguments by claiming `That's just what Hitler would say!' '')  Also applies
    to discounting or rejecting certain arguments, facts and evidence (or even
    experiences!) out of hand because they are supposedly ``against the Bible''
    or other sacred dogma (See also the A Priori Argument). This fallacy
    occasionally degenerates into a separate, distracting argument over who
    gets to define the parameters, tones, dogmas and taboos of the main
    argument, though at this point reasoned discourse most often breaks down
    and the entire affair becomes a naked Argumentum ad Baculum. See also,
    MYOB, Tone Policing, and Calling ``Cards''.

	\subsection{They're All Crooks} The common contemporary fallacy of refusing
	to get involved in public politics because ``all'' politicians and politics
	are allegedly corrupt, ignoring the fact that if this is so in a democratic
	country it is precisely because decent people like you and I refuse to get
	involved, leaving the field open to the ``crooks'' by default. An example
	of Circular Reasoning. Related to this fallacy is ``They're All Biased,''
	the extremely common contemporary cynical fallacy of ignoring news and news
	media because none tells the ``objective truth'' and all push some
	``agenda.''  This basically true observation logically requiring audiences
	to regularly view or read a variety of media sources in order to get any
	approximation of reality, but for many younger people today (2017) it means
	in practice, ``Ignore news, news media and public affairs altogether and
	instead pay attention to something that's fun, exciting or personally
	interesting to you.'' The sinister implication for democracy is, ``Mind
	your own business and leave all the `big' questions to your betters, those
	whose job is to deal with these questions and who are well paid to do so.''
	See also the Third Person Effect, and Deliberate Ignorance.

	\subsection{The ``Third Person Effect'' } (also, ``Wise up!'' and ``They're
	All Liars'')  An example of the fallacy of Deliberate Ignorance, the
	arch{-}cynical postmodern fallacy of deliberately discounting or ignoring
	media information a priori, opting to remain in ignorance rather than
	``listening to the lies'' of the mainstream media, the President, the
	``medical establishment,'' professionals, professors, doctors and the
	``academic elite'' or other authorities or information sources, even about
	urgent subjects (e.g., the need for vaccinations) on which these sources
	are otherwise publicly considered to be generally reliable or relatively
	trustworthy.
	\href{https://www.eurekalert.org/pub_releases/2017-02/du-na020817.php}{According
	to Drexel University researchers} (2017), the ``Third Person Effect ...
	suggests that individuals will perceive a mass media message to have more
	influence on others, than themselves. This perception tends to counteract
	the message's intended `call{-}to{-}action.' Basically, this suggests that
	over time people wised up to the fact that some mass media messages were
	intended to manipulate them {-}{-} so the messages became less and less
	effective.'' This fallacy seems to be opposite to and an overreaction to
	the Big Lie Technique. See also, Deliberate Ignorance, the Simpleton's
	Fallacy, and Trust your Gut.

	\subsection{The ``Thousand Flowers'' Fallacy} (also, ``Take names and kick
	butt.'') A sophisticated, modern ``Argumentum ad Baculum'' in which free
	and open discussion and ``brainstorming'' are temporarily allowed and
	encouraged (even demanded) within an organization or country not primarily
	in order to hear and consider opposing views, but rather to ``smoke out,''
	identify and later punish, fire or liquidate dissenters or those not
	following the Party Line. The name comes from the Thousand Flowers Period
	in Chinese history when Communist leader Chairman Mao Tse Tung applied this
	policy with deadly effect.

	\subsection{Throwing Good Money After Bad} (also, ``Sunk Cost Fallacy'') In
	his excellent book,
	\href{https://www.amazon.com/Logically-Fallacious-Ultimate-Collection-Fallacies/dp/1456624539/}{Logically
	Fallacious} (2015),
	\href{https://www.logicallyfallacious.com/tools/lp/Bo/LogicalFallacies/173/Sunk_Cost_Fallacy}{Author
	Bo Bennett} describes this fallacy as follows ``Reasoning that further
	investment is warranted on the fact that the resources already invested
	will be lost otherwise, not taking into consideration the overall losses
	involved in the further investment.''  In other words, risking additional
	money to ``save'' an earlier, losing investment, ignoring the old axiom
	that ``Doing the same thing and expecting different results is the
	definition of insanity.'' E.g., ``I can't stop betting now, because I
	already bet the rent and lost, and I need to win it back or my wife will
	kill me when I get home!'' See also Argument from Inertia.

	\subsection{TINA} (There Is No Alternative. Also the ``Love it or Leave
	It'' Fallacy; ``Get over it,'' ``Suck it up,'' ``It is what it is,''
	``Actions/Elections have consequences,'' or the ``Fait Accompli'') A very
	common contemporary extension of the either/or fallacy in which someone in
	power quashes critical thought by announcing that there is no realistic
	alternative to a given standpoint, status or action, arbitrarily ruling any
	and all other options out of bounds, or announcing that a decision has been
	made and any further discussion is insubordination, disloyalty, treason,
	disobedience or simply a waste of precious time when there's a job to be
	done. (See also, ``Taboo;'' ``Finish the Job.'')  TINA is most often a
	naked power{-}play, a slightly more sophisticated variety of the Argumentum
	ad Baculum.  See also Appeal to Closure.

	\subsection{Tone Policing}. A corrupt argument from pathos and delivery,
	the fallacy of judging the validity of an argument primarily by its
	emotional tone of delivery, ignoring the reality that a valid fact or
	argument remains valid whether it is offered calmly and deliberatively or
	is shouted in a ``shrill'' or even ``hysterical'' tone, whether carefully
	written and published in professional, academic language in a respected,
	peer{-}reviewed journal or screamed through a bull{-}horn and peppered with
	vulgarity. Conversely, a highly urgent emotional matter is still urgent
	even if argued coldly and rationally.  This fallacy creates a false
	dichotomy between reason and emotion and thus implicitly favors those who
	are not personally involved or emotionally invested in an argument, e.g.,
	``I know you're upset, but I won't discuss it with you until you calm
	down,'' or ``I'd believe what you wrote were it not for your adolescent
	overuse of exclamation points throughout the text.'' Or alternately, ``You
	seem to be taking the death of your spouse way too calmly. You're under
	arrest for homicide. You have the right to remain silent...'' Tone Policing
	is frequent in contemporary discourse of power, particularly in response to
	discourse of protest, and is occasionally used in sexist ways, e.g. the
	accusation of being ``shrill'' is almost always used against women, never
	against men. See also, The F{-}Bomb.

    \subsection{Transfer} (also, Name Dropping) A corrupt argument from ethos,
    falsely associating a famous or respected person, place or thing with an
    unrelated thesis or standpoint (e.g. putting a picture of the Rev. Dr.
    Martin Luther King Jr. on an advertisement for mattresses, using Genghis
    Khan, a Mongol who hated Chinese, as the name of a Chinese restaurant, or
    using the Texas flag to sell more cars or pickups in Texas that were made
    in Detroit, Kansas City or Korea). This fallacy is common in contemporary
    academia in the form of using a profusion of scholarly{-}looking citations
    from respected authorities to lend a false gravitas to otherwise specious
    ideas or text. See also ``Star Power''.

	\subsection{Trust your Gut} (also, Trust your Heart; Trust Your Feelings;
	Trust your Intuition; Trust your Instincts; Emotional Reasoning) A corrupt
	argument from pathos, the ancient fallacy of relying primarily on ``gut
	feelings'' rather than reason or evidence to make decisions. A
	\href{https://www.eurekalert.org/pub_releases/2017-09/osu-ro091817.php}{recent
	(2017) Ohio State University study finds}, unsurprisingly, that people who
	``trust their gut'' are significantly more susceptible to falling for
	``fake news,'' phony conspiracy theories, frauds and scams than those who
	insist on hard evidence or logic. See also Deliberate Ignorance, the
	Affective Fallacy, and The ``Third Person Effect''.

	\subsection{Tu Quoque} (``You Do it Too!''; also, Two Wrongs Make a Right)
	A corrupt argument from ethos, the fallacy of defending a shaky or false
	standpoint or excusing one's own bad action by pointing out that one's
	opponent's acts, ideology or personal character are also open to question,
	or are perhaps even worse than one's own. E.g., ``Sure, we may have
	tortured prisoners and killed kids with drones, but we don't cut off heads
	like they do!'' Or, ``You can't stand there and accuse me of corruption!
	You guys are all into politics and you know what we have to do to get
	reelected!''  Unusual, self{-}deprecating variants on this fallacy are the
	Ego / Nos Quoque Fallacies (``I / we do it too!''), minimizing or defending
	another's evil actions because I am / we are guilty of the same thing  or
	of even worse. E.g., In response to allegations that  Russian Premier
	Vladimir Putin is a ``killer,'' American President Donald Trump (2/2017)
	told an interviewer, ``There are a lot of killers. We've got a lot of
	killers. What, do you think our country's so innocent?''  This fallacy is
	related to the Red Herring and to the Ad Hominem Argument.

	\subsection{Two{-}sides Fallacy} (also, Teach the Controversy) The
	presentation of an issue that makes it seem to have two sides of equal
	weight or significance, when in fact a consensus or much stronger argument
	supports just one side. Also called “false balance” or “false equivalence.”
	(Thanks to
	\href{https://www.tolerance.org/magazine/fall-2017/speaking-of-digital-literacy}{Teaching
	Tolerance} for this definition!)  E.g,. ``Scientists theorize that the
	Earth is a sphere, but there are always two sides to any argument  Others
	believe that the Earth is flat and is perched on the back of a giant
	turtle, and a truly balanced presentation of the issue requires teaching
	both explanations without bias or unduly favoring either side over the
	other''.

	\subsection{Two Truths} (also, Compartmentalization; Epistemically Closed
	Systems; Alternative Truth) A very corrupt and dangerous fallacy of logos
	and ethos, first formally described in medieval times but still common
	today, holding that there exists one ``truth'' in one given environment
	(e.g., in science, work or school) and simultaneously a different, formally
	contradictory but equally true ``truth'' in a different epistemic system,
	context, environment, intended audience or discourse community (e.g., in
	one's religion or at home). This can lead to a situation of stable
	cognitive dissonance where, as UC Irvine scholar Dr. Carter T. Butts
	describes it (2016), ``I know but don't believe,'' making rational
	discussion difficult, painful or impossible. This fallacy also describes
	the discourse of politicians who cynically proclaim one ``truth'' as mere
	``campaign rhetoric'' used ``to mobilize the base,'' or ``for domestic
	consumption only,'' and a quite different and contradictory ``truth'' for
	more general or practical purposes once in office.  See also Disciplinary
	Blinders; Alternative Truth.

	\subsection{Venting} (also, Letting off Steam; Loose Lips) In the Venting
	fallacy a person argues that her/his words are or ought to be exempt from
	criticism or consequence because s/he was ``only venting,'' even though
	this very admission implies that the one ``venting'' was, at long last,
	freely expressing his/her true, heartfelt and uncensored opinion about the
	matter in question. This same fallacy applies to minimizing, denying the
	significance of or excusing other forms of frank, unguarded or uninhibited
	offensive expression as mere ``Locker{-}room Talk,'' ``Alpha{-}male Speech''
	or nothing but cute, adorable, perhaps even sexy ``Bad{-}boy Talk.'' See
	also, the Affective Fallacy. Opposite to this fallacy are the fallacies of
	Political Correctness and the Scripted Message, above.

	\subsection{Venue} The ancient fallacy of Venue, a corrupt argument from
	kairos, falsely and arbitrarily invalidates an otherwise{-}valid argument
	or piece of evidence because it is supposedly offered in the wrong place,
	at the wrong moment or in an inappropriate court, medium or forum.
	According to PhD student Amanda Thran, ``Quite often, people will say to me
	in person that Facebook, Twitter, etc. are `not the right forums' for
	discussing politically and socially sensitive issues. ... In this same
	vein, I’ve also encountered the following argument `Facebook, which is used
	for sharing wedding, baby, and pet photos, is an inappropriate place for
	political discourse; people don’t wished to be burdened with that when they
	log in.' In my experience, this line of reasoning is most often employed
	(and abused) to shut down a conversation when one feels they are losing it.
	Ironically, I have seen it used when the argument has already been
	transpiring on the platform [in] an already lengthy discussion.'' See also
	Disciplinary Blinders.

    \subsection{We Have to Do Something} (also,  the Placebo Effect; Political
    Theater; Security Theater; We have to send a message) The dangerous
    contemporary fallacy that when ``People are scared / People are angry /
    People are fed up / People are hurting / People want change'' it becomes
    necessary to do something, anything, at once without stopping to ask
    ``What?'' or ``Why?'', even if what is done is an overreaction, is a
    completely ineffective sham, an inert placebo, or actually makes the
    situation worse, rather than ``just sitting there doing nothing.'' (E.g.,
    ``Banning air passengers from carrying ham sandwiches onto the plane and
    making parents take off their newborn infants' tiny pink baby{-}shoes
    probably does nothing to deter potential terrorists, but people are scared
    and we have to do something to respond to this crisis!'') This is a badly
    corrupted argument from pathos. (See also ``Scare Tactic'' and ``The Big
    `But' Fallacy.'')

    \subsection{Where there’s Smoke, there’s Fire} (also Hasty Conclusion; Jumping
    to a Conclusion) The dangerous fallacy of ignorantly drawing a snap
    conclusion and/or taking action without sufficient evidence. E.g.,
    “Captain! The guy sitting next to me in coach has dark skin and is reading
    a book in some kind of funny language all full of accent marks, weird
    squiggles above the `N's' and upside{-}down question marks. It must be
    Arabic! Get him off the plane before he blows us all to kingdom come!” A
    variety of the “Just in Case” fallacy. The opposite of this fallacy is the
    ``Paralysis of Analysis.'' 

	\subsection{The Wisdom of the Crowd} (also, The Magic of the Market; the
	Wikipedia Fallacy; Crowdsourcing) A very common contemporary fallacy that
	individuals may be wrong but ``the crowd'' or ``the market'' is infallible,
	ignoring historic examples like witch{-}burning, lynching, and the market
	crash of 2008. This fallacy is why most American colleges and universities
	currently (2017) ban students from using Wikipedia as a serious reference
	source.

	\subsection{The Worst{-}Case Fallacy} (also, ``Just in case;'' ``We can't
	afford to take chances;'' ``An abundance of caution;'' ``Better Safe than
	Sorry;'' ``Better to prevent than to lament.'') A pessimistic fallacy by
	which one’s reasoning is based on an improbable, far{-}fetched or even
	completely imaginary worst{-}case scenario rather than on reality. This
	plays on pathos (fear) rather than reason, and is often politically
	motivated. E.g., ``What if armed terrorists were to attack your county
	grain elevator tomorrow morning at dawn? Are you ready to fight back?
	Better stock up on assault rifles and ammunition today, just in case!''
	See also Scare Tactics. The opposite of this is the Positive Thinking
	Fallacy.

    \subsection{The Worst Negates the Bad} 

	(also, Be Grateful for What You've Got) The extremely common modern logical
	fallacy that an objectively bad situation somehow isn't so bad simply
	because it could have been far worse, or because someone, somewhere has it
	even worse. E.g., ``I cried because I had no shoes, until I saw someone who
	had no feet.'' Or, ``You're protesting because you earn only \$7.25 an
	hour? You could just as easily be out on the street! I happen to know there
	are people in Uttar Pradesh who are doing the very same work you're doing
	for one tenth of what you're making, and they're pathetically glad just to
	have work at all.  You need to shut up, put down that picket sign, get back
	to work for what I care to pay you, and  thank me each and every day for
	giving you a job!'' 

    \subsection{Zero Tolerance} 

	(also, Zero Risk Bias, Broken Windows Policing, Disproportionate Response;
	Even One is Too Many; Exemplary Punishment; Judenrein) The contemporary
	fallacy of declaring an ``emergency'' and promising to disregard justice
	and due process and devote unlimited resources (and occasionally, unlimited
	cruelty) to stamp out a limited, insignificant or even nonexistent problem.
	E.g., ``I just read about an actual case of cannibalism somewhere in this
	country. That's disgusting, and even one case is way, way too many! We need
	a Federal Taskforce against Cannibalism with a million{-}dollar budget and
	offices in every state, a national SCAN program in all the grade schools
	(Stop Cannibalism in America Now!), and an automatic double death penalty
	for cannibals; in other words, zero tolerance for cannibalism in this
	country!'' This is a corrupt and cynical argument from pathos, almost
	always politically driven, a particularly sinister variety of Dog Whistle
	Politics and the ``We Have to do Something'' fallacy. See also, ``Playing
	on Emotions,'' ``Red Herring,'' and also the ``Big Lie Technique''.

\end{multicols}
\end{document}
